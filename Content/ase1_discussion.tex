In this study, we examined how the diffusible MT crosslinker Ase1 affects MT depolymerization, both when connecting MTs and on individual MTs. Our findings indicate that Ase1 reduces MT depolymerizing speeds and selectively enhances rescue frequencies of MTs in antiparallel configurations, thereby stabilizing antiparallel overlaps while having minimal impact on parallel overlaps or isolated MTs. In bipolar MT arrays like the mitotic spindle, this attribute of Ase1 could enable a selective stabilization of the array's central regions, while keeping the rest of the array dynamic and pliable.\par

How does Ase1 binding oppose microtubule depolymerization? 

\begin{itemize}
    \item It is possible that the binding of Ase1 induces changes in the microtubule lattice which stabilize it against depolymerization. Indeed, it has been found that microtubules decorated with Ase1 have a lower flexural rigidity than undecorated microtubules \parencite{Portran2013}. \cite{Portran2013} had speculated that this may indicate that Ase1 may introduce structural changes to the microtubule lattice. However, no such changes were reported in the cryo-EM study by \cite{Kellogg2016}, though such potential changes seemingly were not the focus of that study. It bears noting that it appears unlikely that Ase1 stabilizes the (GDP) microtubule lattice in a similar manner as Kinesin-1, which expands the lattice, thereby bringing it into a more GTP-like state \parencite{Peet2018}. This is because we observed Ase1 to preferably bind to GDP-lattices \pref{ase2b}{A}, indicating that it prefers a compacted lattice.
    \item The putative protofilament-bridging characteristic of Ase1 \pref{sec:Ase1_intro}{} could plausibly cause the microtubule-stabilizing character of Ase1 as reported here. Namely, bridging protofilaments could oppose depolymerization by tethering protofilaments together, thereby preventing them from curling outward \pref{sec:Ase1_intro}{}. Such a potential stabilization due to the bridging of protofilaments could thus be understood as supporting the role of the microtubule lattice in straightening protofilaments.
\end{itemize}
Microtubule growth likely proceeds by incorporating slightly outward-bent protofilaments into the lattice while straightening them \pcite{Cross2019}. Thus, one may expect Ase1 to also increase microtubule polymerization rates if either of the above explanations is correct, as the stabilization of lateral tubulin bonds is hypothesized to be the mechanism by which the EB protein accelerates microtubules growth \pcite{Gudimchuk2021}. The fact that we did not observe such an increase can be explained by our observation that Ase1 prefers to bind to GDP-lattices, i.e., not the GTP-dominated lattice at the tip of growing microtubules (contrary to what is the case for EB proteins).

\par
What causes the markedly-pronounced stabilization of antiparallel microtubules in particular? Four potential mechanisms appear plausible.

\begin{itemize}
    \item Within antiparallel overlaps, Ase1 diffusion constant and unbinding rates are greatly reduced \parencite{Kapitein2008, lanskydiffusible2015}, partly due to protein avidity resulting from the multivalent interactions of Ase1 with the MTs \parencite{braun2020cytoskeletal, erlendsson2021binding}. In other words, Ase1 molecules within antiparallel overlaps are less likely to unbind from a given binding site, and hence they can counteract microtubule depolymerization (via one or both of the mechanisms proposed above) for a prolonged timeframe when bound to a terminal tubulin dimer.
    \item Depolymerizing involves a bending of the tubulin subunits at the microtubule tip \pref{sec:instability}. However, precisely this bending might be opposed by the other microtubule, as an outward-bending protofilament has to push against it. Under the absence of Ase1 in solution, this is not an issue (as we had shown in \aref{ase1c}{}), likely because the other microtubule can easily move away from the tip of the depolymerizing microtubule. However, it seems likely that such a movement would require more energy in the case of a crosslinked microtubule. Not only does it appear likely that the terminal Ase1 would oppose such a separation, but also all the other crosslinking Ase1 molecules in the vicinity of the tip, potentially resulting in a more stable microtubule tip structure allowing for regaining a GTP cap. One may expect this mechanism to apply to parallel overlaps as well. However, given the low affinity of Ase1 molecules to parallel overlaps, it appears plausible that Ase1 binds microtubules in a parallel conformation too weakly to make microtubule unzipping costly enough for it to noticably hamper microtubule depolymerization.
    \item Another potential mechanism increasing the stabilizing effect of Ase1 in the case of antiparallel overlaps could be the multimerization of Ase1 within antiparallel overlaps as reported by \cite{Kapitein2008}, a feature recently shown to play a crucial role in slowing motor-driven MT sliding \parencite{alfieri2021two}. Ase1 multimers could enhance the (potential) capability of Ase1 to bridge multiple protofilaments.
    \item Lastly, crosslinking Ase1 could increase rescue frequency not due to its presence at the depolymerizing tip but by a putative mechanism based on GTP island creation. Here, the (antiparallel) crosslinking activity of Ase1 would increase the likelihood of lattice defects being incorporated into the lattice of a growing microtubule. For example, at the growing tip, a crosslink could potentially ocassionally be established between one of the growing protofilaments and the template microtubule, introducing conformational constraints which might lead to lattice defects. When later repaired by incorporation of free GTP-tubulin, such defects could result in the emergence of GTP islands and hence potential locations where rescues are more likely \pref{sec:instability}{}. This mechanism however cannot account for a correlation between increased rescue frequency at the tip and the number of herded Ase1 molecules at the depolymerizing tip.
\end{itemize}
The latter three of these proposed mechanisms could potentially account for the qualitative difference we observed between antiparallel and isolated microtubules, namely that Ase1 promoted rescues only in the case of antiparallel microtubules.\par

In addition to having observed a selective stabilization of antiparallelly crosslinked microtubules, we also observed herding of Ase1 by depolymerizing microtubule ends. We produced a simple model based on the assumption that Ase1 reduces the detachment of the terminal tubulin subunit when bound at the tip (only modeling one protofilament, neglecting potential protofilament interactions). This assumption, when allowing for diffusion of Ase1 molecules along the protofilament, leads to both a decrease of MT depolymerization velocity and accumulation of Ase1 at the tip of shrinking MTs. The same had been found in a parallel effort by \cite{Hiyasat}, who modeled the accumulation of spastin at the depolymerizing mt end and the concomitant decrease in depolymerization velocity.\par

For isolated microtubules, our model's fit with our data increased substantially when we introduced the assumption that an Ase1 molecule bound to a terminal tubulin dimer also prevents the dimers on the two neighbouring protofilaments from depolymerizing. This supports the notion that Ase1 might bridge protofilaments via its C-terminal. Alternatively, or in addition, one can also envision an indirect stabilization through lateral protofilament interactions. This model quantitatively recapitulated the behavior of the system for 6 nM Ase1, and within an order of magnitude for 1 nM Ase1. Given the low density of Ase1 molecules at 1 nM concentration (<1\% of tubulin dimers bound to Ase1), the discrepancy may be due to stochasticity of the system. There was one more potential discrepancy between our modeling results and our experimental results for isolated MTs, namely that the model predicted a characteristic length $\lambda$ of the exponential decay of Ase1 at the microtubule end of around 600nm, while we measured only around 200nm. However, this signal is hardly comparable with our isolated-protofilament model, not only because it comes from multiple protofilaments that may not be in register, but also because shrinking protofilaments are likely curved outwards \parencite{McIntosh2008}. \par

What could explain the quantitative disagreement with experimental data for antiparallelly crosslinked MTs? In principle, the fact that a disagreement exists is not surprising, given that overlaps are not symmetric, and some protofilaments have almost no Ase1, while others have extremely high Ase1 density (76\% tubulin dimers bound to Ase1 in the MT body assuming 2 protofilaments crosslinked, 50\% assuming 3, see “Mathematical modelling” in Methods). It is for instance conceivable that the crosslinked protofilements lag behind the non-crosslinked protofilements, similar to what had been observed by \cite{Peet2018} when binding depolymerizing microtubules to the coverslip surface via kinesin-1. Hence, a more complex model accounting for protofilament interactions would be needed for overlaps. Such a model would likely need to be informed by experimental measurements of such interactions. However, it is also possible that we simply are overestimating the number of Ase1 molecules herded by antiparallel MTs, because part of the Ase1 which is lost at the depolymerizing MT end presumably remains bound to the template MT, an effect which we do not account for in our estimation. A likely expression of this effect is our observation that while at the ends of isolated MTs, we observed a (blurred) right-sided exponential decay of additional Ase1 density as predicted by our model, the additional Ase1 density at the ends of antiparallel MTs were more reminiscent of a gaussian \pref{ase2e}{B-D}. In particular, the exponential fits often did not fully capture the additional Ase1 density we observed which "lingered" behind the depolymerizing MT end, an additional density which the gaussian fits did capture (notably, because this additional density is likely due to Ase1 molecules still bound to the template MT after detaching from the depolymerizing MT end, we decided to base our analysis as shown in \aref{ase2d}{} on the results stemming from the exponential fits).
Our model did not include MT rescues; however, if one assumes that each crosslink reached by a depolymerizing MT tip has a chance of inducing rescue, as proposed by \cite{Stoppin-Mellet2013}, we expect a positive correlation between Ase1 density and rescue frequency, consistent with our experimental data for antiparallel overlaps. Indeed, the relationship we observed appeared to be a linear one, which would precisely be the relationship proposed by \cite{Stoppin-Mellet2013}.\par 

It also bears noting that protein herding at the depolymerizing microtubule tip has been observed for kinetochore-associated Ndc80 and Dam1 complexes, which crosslink chromosomes to depolymerizing MT ends \pcite{Westermann2006, Tooley2011}. Herding of these proteins is thus essential for mitosis. As with Ase1, herding of Dam1- and Ndc80- complexes at depolymerizing MT ends hampers the dissociation of tubulin subunits from these MT ends \parencite{Franck2007, umbreit2012ndc80, grishchuk2017biophysics}. However, the word "herding" does not suggest itself for these complexes, because the number of "herded" proteins is much lower than in the Ase1 or spastin cases. Another difference is that these complexes rarely dissociate from the depolymerizing microtubule, presumably due to their high multivalency \pref{volkov2018multivalency}{} and formation of ring-structures around the microtubule \pcite{Westermann2006}. Nonetheless, given the many similarities to our case study, exploring the literature on this phenomenon is instructive. \cite{grishchuk2017biophysics} use the term "biased diffusion" to describe a potential mechanism for the herding of Dam1/Ndc80 which essentially is the same as the proposed mechanism underlying our model. They contrast this with a second potential mechanism, which is based on protofilament "powerstrokes," i.e., the fact that protofilaments generate forces when bending outward during depolymerization \pref{sec:instability}{}. In this model, the biased movement at the depolymerizing tip, i.e., the herding of Dam1/Ndc80, would stem from such forces. Given that Ndc80 and Dam1 are capable of transducing large forces, it is likely that they indeed harvest protofilament powerstrokes \pcite{grishchuk2017biophysics}. In this context, it is interesting to note that we found that accumulated Ase1 can transduce forces to other MTs \pref{ase2c}{B}. Notably, starPEG-(KA7)4, a synthetic MT crosslinker with multivalent MT-binding interfaces has recently been shown to also drag MTs when being swept by depolymerizing MTs \parencite{Drechsler2019}, even though it did not hinder MT depolymerization of isolated MTs, pointing toward a potential powerstroke mechanism. It may be interesting to measure the forces which Ase1 sweeping is capable of transducing, which would also shed light on the question on whether protofilament powerstrokes are an important component of Ase1 sweeping (biased diffusion can also generate forces, see \cite{grishchuk2017biophysics}, which can be understood as entropic forces \cite{lanskydiffusible2015}). Interestingly, in the case of Dam1- and Ndc80- complexes, it has been shown that exerting load on the complexes, i.e., pulling them against the direction of microtubule depolymerization, increases their \textit{in vitro} rescue-promoting activity \parencite{Franck2007, volkov2018multivalency}. This indicates that opposing potential protofilament powerstrokes may induce rescues, and hence lends support to the idea that crosslinking Ase1 promotes rescues by potentially opposing the bending of protofilaments. \par

Our model of the mechanisms behind Ase1 sweeping moreover suggests that any diffusing molecule that prevents tubulin unbinding will "track" depolymerizing ends, even if it may not be propelled by protofilament powerstrokes. For Ase1 specifically, our finding that Ase1 accumulates at the ends of depolymerizing plus ends may imply interesting biological properties, as it may have relevance for the localization of the MAPs which Ase1 recruits to the microtubule \pref{sec:Ase1_intro}{}. For instance, a high density of Ase1 molecules within a largely-shrunk antiparallel overlap might thus imply a larger recruitment activity for the rescue factor CLASP, ensuring that overlaps do not vanish completely. \par
 
Our results show that the presence of diffusible MT crosslinkers can suffice to establish enduring antiparallel MT overlaps. Antiparallel Mt overlaps are found in the midzone of mitotic spindles, however, as an important caveat, the biological significance of our findings is unclear. Regarding the mitotic spindle of the fission yeast S. pombe in particular, \cite{Bratman2007b} had reported that the microtubule rescue factor CLASP was necessary to stabilize the antiparallel overlap regions against disassmbly via microtubule depolymerization. When removing the C-terminal of Ase1, Ase1 no longer recruited CLASP but still partitioned into antiparallel overlaps, yet the mitotic spindles of cells expressing this Ase1$\delta$C construct had a similar number of deformed mitotic spindles as the spindles of cells which did not express Ase1 at all. While this may seem to directly contradict our findings, it can be noted that the C-terminal of Ase1 not only is important for recruiting CLASP, but also has other regulatory functions potentially relevant in the given context (it should be noted that \cite{Bratman2007b} did not truncate the whole C-terminal, in particular, their construct retained the residues crucial for nuclear localization). For instance, the C-terminal region has been found to recruit klp9p, a kinesin-6 motor promoting spindle elongation \pcite{fu2009phospho}. Given that the interplay of Ase1 and motor proteins controls spindle positioning \pcite{Braun2011}, it appears possible that the deletion of the C-terminal region may negatively impact spindle structure also via the failure of such positioning mechanisms. Moreover, given that the C-terminal of Ase1 likely bridges protofilaments \pcite{Kellogg2016}, it may be essential for its microtubule-stabilizing character as reported here. Future experiments could thus investigate whether our findings also hold for an Ase1 construct lacking the C-terminal. Lastly, while CLASP is a stronger and more vital microtubule rescue factor than Ase1 in fission yeast, microtubule stabilization by Ase1 may still play a supporting role, and potentially a more important role in other organisms. \par

An earlier study on a plant Ase1 analogue, MAP65-1, found increased rescue rates of MTs within bundles compared to isolated MTs \parencite{Stoppin-Mellet2013}. This study, however, did not experimentally distinguish between parallel and antiparallel bundles. Our methods allowed us to directly distinguish between different bundling orientations, and our findings in fact support the modeling-based findings by \parencite{Stoppin-Mellet2013}. However, our results do not rule out that under different experimental conditions, Ase1 may also stabilize parallel bundles to some degree.\par

As with tau, it also in the present case is tempting to speculate that the impact of diffusible crosslinkers on MT dynamics may be tunable by posttranslational modifications of either the crosslinkers or the MT surface. This could give the cell spatial, and more importantly, temporal control of the stability of (mitotic) microtubule arrays, e.g., given that the phosphorylation state of Ase1 changes during mitosis \pcite{Khmelinskii2009}. \par

Beyond the mitotic spindle and other microtubule systems featuring diffusive microtubule crosslinkers of the Ase1/MAP65/PRC1 family, our findings also could hint at a more general principle: For actin filament overlaps, it has been observed that F-actin crosslinkers slow down actin depolymerization \parencite{maul2003eplin,schmoller2011slow}, suggesting that crosslinker-dependent stabilization of filaments may be a fundamental mechanism, widespread across cytoskeletal systems.