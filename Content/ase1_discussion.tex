In this study, we examined how the diffusible MT crosslinker Ase1 affects MT depolymerization, both when connecting MTs and on individual MTs. Our findings indicate that Ase1 reduces MT depolymerizing speeds and selectively enhances rescue frequencies of MTs in antiparallel configurations, thereby stabilizing antiparallel overlaps while having minimal impact on parallel overlaps or isolated MTs. In bipolar MT arrays, like mitotic spindles, this attribute of Ase1 could enable a selective stabilization of the array's central regions, while keeping the rest of the array dynamic and pliable. An earlier study on a plant Ase1 analogue, MAP65-1, found increased rescue rates of MTs within bundles compared to isolated MTs \parencite{Stoppin-Mellet2013}. This study, however, did not experimentally distinguish between parallel and antiparallel bundles. Our methods allowed us to directly distinguish between different bundling orientations, and our findings in fact support the modeling-based findings by \parencite{Stoppin-Mellet2013}. However, our results do not rule out that under different experimental conditions, Ase1 may also stabilize parallel bundles to some degree.

We observed Ase1 sweeping, the accumulation of lattice-bound, diffusible Ase1 at the retracting ends of depolymerizing MTs, and found that accumulated Ase1 can transduce forces to other MTs. This phenomenon, also known as "protein sweeping" or "herding," is analogous to the in vitro behavior of the MT-severing enzyme spastin and the kinetochore-associated Ndc80 and Dam1 complexes, which crosslink chromosomes to depolymerizing MT ends \parencite{Franck2007, umbreit2012ndc80, grishchuk2017biophysics}, indicate that, analogously to Dam1- and Ndc80- complexes, Ase1 accumulation at depolymerizing MT ends, antagonizes the dissociation of tubulin subunits from these MT ends. While on isolated MTs this effect may be neutralized, either by Ase1 dissociation or translocation along the MT lattice, both of these processes happen less readily within overlaps, where Ase1 diffusion constant and unbinding rates are greatly reduced \parencite{Kapitein2008, lanskydiffusible2015}, due to protein avidity resulting from the multivalent interactions of Ase1 with the MTs \parencite{braun2020cytoskeletal, erlendsson2021binding}. Additionally, in overlaps, individual protofilaments of a depolymerizing MT are coupled to the other MT by the Ase1 crosslinkers and in order to bend into the rams horn formations associated with depolymerization, likely have to work against the lattice of the other MT, which might lead to further stabilization. This effect is analogous to Dam1- and Ndc80- complexes whose rescue-promoting propensity can be enhanced by exerting load on the complexes \parencite{Franck2007, volkov2018multivalency}. Importantly, this scenario applies only to antiparallel MT overlaps, as for parallel overlaps no stabilizing effect additional to the one observed on isolated MTs arises, indicating that there the force-coupling is weak. This is consistent with the fact that Ase1 binds with lower affinity to parallel, compared to antiparallel MTs. Nevertheless, the mere presence of Ase1 on the lattice of isolated MTs is sufficient to reduce depolymerization velocities. Our modeling suggests that Ase1 may stabilize adjacent protofilaments, which might be explained by Ase1 binding to a tubulin dimer allosterically stabilizing the adjacent protofilament either by Ase1's intrinsically disordered N-terminus directly binding to the adjacent protofilament \parencite{Subramanian2010} or indirectly through the tubulin lattice, such as kinesin-1 \parencite{Peet2018, morikawa2015x}. 

Our modelling further suggests that the ability of Ase1 to both diffuse and reduce tubulin subunit detachment at depolymerizing plus ends confers interesting biological properties. Not only can Ase1 reduce the depolymerization speed of MTs, but this makes Ase1 capable of tracking depolymerizing ends, since subunits without Ase1 are more likely to be lost. Our findings thus suggest that any diffusing molecule that prevents tubulin unbinding will track depolymerizing ends, and therefore may exert forces on objects that the molecule has affinity for on accessible regions as the MT shrinks. Conversely, these forces will drag the molecule in the opposite direction of MT depolymerization, making it more likely to be at the terminal subunits, amplifying its braking effect on depolymerization speed. It may be interesting to measure the forces which Ase1 sweeping is capable of transducing, which would also shed light on the question on whether protofilament powerstrokes are an important component of Ase1 sweeping. Interestingly, starPEG-(KA7)4, a synthetic MT crosslinker with multivalent MT-binding interfaces has recently been shown to also drag MTs when being swept by depolymerizing MTs \parencite{Drechsler2019}, even though it did not hinder MT depolymerization of isolated MTs. It would be interesting to scrutinize the dynamics of bundles crosslinked by starPEG-(KA7)4 in future studies, as well as the depolymerization speed of "pulling" MTs. 

We produced simple models based on the assumption that Ase1 reduces the detachment of terminal tubulin subunits when bound at the MT tip. This assumption, when allowing for diffusion of Ase1 molecules along the protofilament, leads to both a decrease of MT depolymerization velocity and accumulation of Ase1 at the tip of shrinking MTs. Our model quantitatively recapitulated the behavior of the system for 6 nM Ase1, and within an order of magnitude for 1 nM Ase1. Given the low density of Ase1 molecules at 1 nM concentration (<1\% of tubulin dimers bound to Ase1), the discrepancy may be due to stochasticity of the system. There was one more potential discrepancy between our modeling results and our experimental results for isolated MTs, namely that the model predicted a characteristic length $\lambda$ of the exponential decay of Ase1 at the microtubule end of around 600nm, while we measured only around 200nm. However, this signal is hardly comparable with our isolated-protofilament model, not only because it comes from multiple protofilaments that may not be in register, but also because shrinking protofilaments are likely curved outwards \parencite{McIntosh2008}. What could explain the quantitative disagreement with experimental data for antiparallelly crosslinked MTs? In principle, the fact that a disagreement exists is not surprising, given that overlaps are not symmetric, and some protofilaments have almost no Ase1, while others have extremely high Ase1 density (76\% tubulin dimers bound to Ase1 in the MT body assuming 2 protofilaments crosslinked, 50\% assuming 3, see “Mathematical modelling” in Methods). It is for instance conceivable that the crosslinked protofilements lag behind the non-crosslinked protofilements, forming a "tail" (TODO: cite paper that shows such a lagging tail). Hence, a more complex model accounting for protofilament interactions would be needed for overlaps. Such a model would likely need to be informed by experimental measurements of such interactions. However, it is also possible that we simply are overestimating the number of Ase1 molecules herded by antiparallel MTs, because part of the Ase1 which is lost at the depolymerizing MT end presumably remains bound to the template MT, an effect which we do not account for in our estimation. A likely expression of this effect is our observation that while at the ends of isolated MTs, we observed a (blurred) right-sided exponential decay of additional Ase1 density as predicted by our model, the additional Ase1 density at the ends of antiparallel MTs were more reminiscent of a gaussian \pref{ase2e}{B-D}. In particular, the exponential fits often did not fully capture the additional Ase1 density we observed which "lingered" behind the depolymerizing MT end, an additional density which the gaussian fits did capture (notably, because this additional density is likely due to Ase1 molecules still bound to the template MT after detaching from the depolymerizing MT end, we decided to base our analysis as shown in \aref{ase2d}{} on the results stemming from the exponential fits).
Our model did not include MT rescues; however, if one assumes that each crosslink reached by a depolymerizing MT tip has a chance of inducing rescue, as proposed by \cite{Stoppin-Mellet2013}, we expect a positive correlation between Ase1 density and rescue frequency, consistent with our experimental data. Indeed, the relationship we observed appeared to be a linear one, which would precisely be the relationship proposed by \cite{Stoppin-Mellet2013}.\par 

Why did we not observe Ase1 to promote rescues on isolated MTs? It appears likely that this stems from the conformational constraints introduced by the microtubule the depolymerizing microtubule is crosslinked to. Depolymerizing involves a bending of the tubulin subunits at the microtubule tip. However, precisely this bending might be opposed by the other microtubule, as an outward-bending protofilament has to push against it. Under the absence of Ase1 in solution, this is not an issue (as we had shown in \aref{ase1c}{}), because the other microtubule can easily move away from the tip of the depolymerizing microtubule. However, it seems likely that such a movement would require more energy in the case of a crosslinked microtubule. Not only does it appear likely that the terminal Ase1 would oppose such a separation, but also all the other crosslinking Ase1 molecules in the vicinity of the tip. Another potential mechanism increasing the stabilizing effect of Ase1 in the case of antiparallel overlaps could be the multimerization of Ase1 within antiparallel overlaps as reported by \cite{Kapitein2008}, a feature recently shown to play a crucial role in slowing motor-driven MT sliding \parencite{alfieri2021two}. This multimerization, particularly when enhanced by Ase1 herding at depolymerizing MT ends, could introduce additional complexity to Ase1-mediated MT dynamics regulation. The possibility of Ase1 molecules acting cooperatively to promote rescues for antiparallel MTs specifically is intriguing and would offer the cell a lever for modulating the rescue-promoting effect of MT crosslinking. \par
 
Our results show that the presence of diffusible MT crosslinkers can suffice to establish enduring antiparallel MT overlaps, such as those found in the midzone of mitotic spindles. In such context Ase1 can work cooperatively with other MT rescue factors such as CLASP \parencite{Bratman2007b} or provide an alternative mechanism for selective life time enhancement of antiparallel overlaps. We speculate that the impact of diffusible crosslinkers on MT dynamics may be tunable by posttranslational modifications of either the crosslinkers or the MT surface. Such a tunability has recently been proposed for a seemingly related capacity of Ase1, namely the braking of MT sliding caused by molecular motors \parencite{fu2009phospho, Thomas2020}. For actin filament overlaps, it has been observed that F-actin crosslinkers slow down actin depolymerization \parencite{maul2003eplin,schmoller2011slow}, suggesting that crosslinker dependent stabilization of filaments may be a fundamental mechanism, widespread across cytoskeletal systems.
