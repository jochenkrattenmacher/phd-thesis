\chapter{Aims of this thesis}
\label{aims}
The aims of this thesis are to further our understanding blablabla

\section{Organization of MT arrays}
The fission yeast Schizosaccharomyces pombe serves as a simple model cell for studying MT dynamics and organization. These cells exhibit two types of structures that contain bundles of stable overlapping anti-parallel MTs: the interphase MT bundles and the mitotic spindle. One important factor for assembly and maintenance of these overlapping MT arrays is a conserved, diffusive MT-bundling protein, ase1p (Loiodice et al., 2005; Yamashita et al., 2005). Ase1p, and its human (PRC1) and plant (MAP-65) orthologues, beside crosslinking MTs, is involved in the regulation of spindle elongation and serves as a complex regulatory platform for the recruitment of other midzone proteins at the spindle midzone (She et al. 2019). Ase1$\Delta$mutants thus, while being viable, exhibit interphase MTs with reduced bundling and mitotic spindles that often fall apart in anaphase. 
An important characteristic common to the Ase1/MAP65/PRC1 family is autonomous and preferential binding to MTs bundled in antiparallel fashion (She et al. 2019). This allows for the precise recruitment of other midzone proteins, but it is also known to have direct effects microtubule dynamics.  In-vitro experiments have shown that MAP65-1, when bundling microtubules, promotes microtubule rescues (Stoppin-Mellet et al. 2013). However, it still is unclear whether, and in how far, microtubules bundled in parallel fashion are affected differently than microtubules bundled in antiparallel fashion. 
We here present the results of in-vitro assays which show that the depolymerization of parallel is indeed affected differently by Ase1 than those of antiparallel MTs. In our assays, we for the first time also observed Ase1 to directly have an impact on the depolymerization of single MTs. 

\section{Elucidation of how tau can affect MTs in its manifold ways}
However, the underlying molecular mechanism of how tau as an intrinsically disordered protein can fulfill such regulative functions is still unknown. Cite: Many faces of tau