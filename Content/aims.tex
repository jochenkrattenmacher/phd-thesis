\chapter{Aim}
The aim of this thesis and the work underlying it is to further our understanding of the ways in which cytoskeletal polymers, in particular microtubules and microtubule-associated proteins (MAPs) interact to give rise to emergent phenomena. To achieve this, we sought to reconstitute microtubule-based cytoskeletal systems \textit{in vitro} (i.e., outside of biological cells) with a minimal variety of components. This approach can help elucidate which components are needed at a minimum for a given phenomenon to emerge. Moreover, given the low number of confounding factors in such relatively simple assays, it often is possible to gain insights into biophysical mechanisms underlying a given observation. \par
We focused on two different MAPs, aiming to elucidate and potentially recreate phenomena which others had previously observed in the case of \textit{in vivo} cytoskeletal systems in which these MAPs are prominently featured. In both cases, we have a strong focus on the distribution of these MAPs on microtubules and the effects of this distribution, aiming to contribute to our understanding of how these MAPs may interact with their environment to help spatially organize microtubules within cells. Importantly however, these two MAPs, namely Ase1 and tau (see \autoref{sec:MAPs}), are found in distinct cellular contexts. While Ase1 and its orthologues are primarily involved in mitotic spindle dynamics and cytokinesis, tau is predominantly functioning in the stabilization of axonal microtubules in neurons. Given that they operate under different cellular environments, we have studied these MAPs separately. This separation also reflects the structure of the research publications that form the basis of this thesis, one of which focusing on tau, the other on Ase1 (see \autoref{chap:publications}). To facilitate a clear presentation of the findings presented in these two publications, I have structured the results and the discussion of this thesis accordingly, i.e., these sections each feature two separate subsections.