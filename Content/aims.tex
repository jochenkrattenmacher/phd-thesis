\chapter{Aim}
The aim of this thesis and the work underlying it is to further our understanding of the ways in which cytoskeletal polymers, in particular microtubules and microtubule-associated proteins (MAPs) interact to give rise to emergent phenomena. To achieve this, we sought to reconstitute microtubule-based cytoskeletal systems \textit{in vitro} (i.e., outside of biological cells) with a minimal variety of components. This approach can help elucidate which components are needed at a minimum for a given phenomenon to emerge. Moreover, given the low number of confounding factors in such relatively simple assays, it often is possible to gain insights into biophysical mechanisms underlying a given observation. \par
We focused on two different MAPs, aiming to elucidate and potentially recreate phenomena which others had previously observed in the case of \textit{in vivo} cytoskeletal systems in which these MAPs are prominently featured. Importantly, these two MAPs, namely Ase1 and tau (see \autoref{sec:MAPs}), are found in distinct cellular contexts. While Ase1 and its homologues in animal and plant cells are primarily involved in mitotic spindle dynamics and cytokinesis, tau is predominantly functioning in the stabilization of axonal microtubules in neurons. Given that they operate under different cellular environments, we have studied these MAPs separately. This separation also reflects the structure of the research publications (see \autoref{chap:publications}) that form the basis of this thesis:
\begin{itemize}
    \item In the first publication \parencite{Siahaan2019a}, titled "Kinetically distinct phases of tau on microtubules regulate kinesin motors and severing enzymes," we share our discovery that there exist two distinct modes of tau-microtubule interaction which result in distinct phases of tau on microtubules. We furthermore show that these phases differ in their interaction with microtubule-associated motor proteins and the microtubule severing enzyme katanin. This research thus has contributed to our understanding of how tau may perform its manifold different regulatory functions (\cite{Morris2011b} in particular have emphasized this notable versatility of tau).
    \item The second publication \parencite{Krattenmacher2024}, "Ase1 selectively increases the lifetime of antiparallel microtubule overlaps," adds to our understanding of how Ase1 affects microtubule dynamic instability (see \autoref{sec:instability}). In particular, our research shows that Ase1, without the help of any other MAP, can give rise to long-lasting antiparallel microtubule overlaps, which are structurally critical features of mitotic spindles.
\end{itemize}
To facilitate a clear presentation of the findings presented in these two publications, I have structured the Results section of this thesis accordingly, i.e., it features two separate subsections. Nonetheless, I with this thesis also aim to connect these two parts of my work to the degree to which this is sensible and insightful. This ambition is reflected in the Discussion section, where I aimed to draw insights which go beyond the pecuilarities of each specific cytoskeletal system.


\section{Organization of MT arrays}
The fission yeast Schizosaccharomyces pombe serves as a simple model cell for studying MT dynamics and organization. These cells exhibit two types of structures that contain bundles of stable overlapping anti-parallel MTs: the interphase MT bundles and the mitotic spindle. One important factor for assembly and maintenance of these overlapping MT arrays is a conserved, diffusive MT-bundling protein, ase1p (Loiodice et al., 2005; Yamashita et al., 2005). Ase1p, and its human (PRC1) and plant (MAP-65) orthologues, beside crosslinking MTs, is involved in the regulation of spindle elongation and serves as a complex regulatory platform for the recruitment of other midzone proteins at the spindle midzone (She et al. 2019). Ase1$\Delta$mutants thus, while being viable, exhibit interphase MTs with reduced bundling and mitotic spindles that often fall apart in anaphase. 
An important characteristic common to the Ase1/MAP65/PRC1 family is autonomous and preferential binding to MTs bundled in antiparallel fashion (She et al. 2019). This allows for the precise recruitment of other midzone proteins, but it is also known to have direct effects microtubule dynamics.  In-vitro experiments have shown that MAP65-1, when bundling microtubules, promotes microtubule rescues (Stoppin-Mellet et al. 2013). However, it still is unclear whether, and in how far, microtubules bundled in parallel fashion are affected differently than microtubules bundled in antiparallel fashion. 
We here present the results of in-vitro assays which show that the depolymerization of parallel is indeed affected differently by Ase1 than those of antiparallel MTs. In our assays, we for the first time also observed Ase1 to directly have an impact on the depolymerization of single MTs. 

\section{Elucidation of how tau can affect MTs in its manifold ways}
However, the underlying molecular mechanism of how tau as an intrinsically disordered protein can fulfill such regulative functions is still unknown. Cite: Many faces of tau