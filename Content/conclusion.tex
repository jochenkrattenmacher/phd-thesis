The results presented in this thesis and the two associated publications \pcite{Krattenmacher2024,Siahaan2019a} shine further light on how the structure of the cytoskeleton can emerge from the collective behavior of its constitutive parts. In particular, this work highlights potential mechanisms for the stabilization and maintenance of key microtubule structures, namely axonal microtubule arrays and antiparallel microtubule arrays in the mitotic spindle. To probe for these potential mechanisms, we had established experiments with a minimal number of components such to exclude other potential factors, which allows us to establish that the phenomenona we observed truly emerged solely from the interaction patterns of these components rather than third factors.\par

In the case of axonal microtubule structures, we focused on the intrinsically disordered protein Tau as well as its interaction patterns with the molecular motor kinesin-8 and the microtubule-severing protein Katanin. Our results highlight the existence of two distinct Tau binding modes: Rapid diffusion along the microtubule as well as stationary, cooperative binding to the microtubule. The cooperative mode involves the formation of Tau islands which grow and shrink from their boundaries. Our findings regarding these Tau islands, in particular their characteristic density of 0.26 Tau molecules per tubulin dimer, support a recently-proposed model of Tau binding, where each Tau molecule binds along the crest of a given protofilament, with each microtubule binding repeat of Tau stretching the length of one tubulin dimer. The fact that we did not observe Tau islands within microtubule bends, but yet observed Tau stationarily bound to such bends, suggested that this stationary binding mode of Tau relies on a specific lattice spacing of the tubulin subunits of microtubules, which was confirmed in later experiments. We found that Tau islands are highly effective at shielding microtubules from severing by Katanin, which contributes to our understanding of how axonal microtubule arrays can be stabilized. Finally, our finding that the molecular motor Kip3 is regulated by Tau islands, and can vice versa can regulate Tau islands, hints at the existence of complex interaction dynamics of microtubule-associated proteins within neuronal microtubule arrays, giving rise to complex patterns in time and/or space.\par

In the case of antiparallel microtubule arrays during mitosis, we focused on the interaction patterns of the microtubule crosslinker Ase1 and dynamic microtubules. Our results demonstrate that Ase1 selectively stabilizes antiparallel microtubule overlaps by both reducing depolymerization rates and enhancing rescue frequencies, while having minimal effect on isolated or parallel microtubules. This selective stabilization likely stems from multiple mechanisms, including the enhanced affinity of Ase1 for antiparallel overlaps and the crosslinking activity of Ase1 itself, thereby preventing their outward curling during depolymerization. We have also found that Ase1, similarly as the microtubule-severing enzyme spastin, is being herded by depolymerizing microtubule tips, and that this herding correlates with a slow-down in microtubule depolymerization. Our mathematical model suggests that this herding effect can be explained solely by Ase1 hampering the dissociation of terminal tubulin dimers to which it is bound. In other words, Ase1 accumulation at the depolymerizing tip does not rely on any physiochemical preference of Ase1 for tubulin dimers at the tip, nor does it rely on protofilament powerstrokes moving the molecules in the direction of depolymerization. The selective stabilization of antiparallel microtubules of Ase1 could help explain the stability of antiparallel microtubule arrays in the mitotic spindle and other antiparallel microtubule arrays, while Ase1 herding could have implications for the localization of the proteins it recruits to the microtubule.\par

Overall, our findings highlight further ways in which microtubules, rather than being static tubes, are physically modifiable structures. This, likely plays an important role in enabling microtubules, in their interaction dynamics with other proteins, to fulfill their diverse roles in the establishment and maintenance of biological patterns.