\chapter{Discussion}
\section{Tau}
\begin{figure}[h!tb]
\centering
\includegraphics[scale=1]{Figures/tau8.png}
\caption[Schematic representation of island formation.]{
\textbf{Schematic representation of island formation.} Tau molecules bind and unbind with high rates to microtubules, on which they diffuse (fast turnover). When encountering an island (dashed orange box), tau molecules cooperatively associate with the island at its boundaries, rendering the tau molecules stationary, decreasing their unbinding rate (slow turnover), and causing the island to grow in size laterally. Tau molecules from solution can only bind to the inside of an island via displacement of an island-associated tau molecule, resulting in the observed concentration-dependent turnover of tau inside islands. After removal of tau from solution, tau molecules dissociate from the island boundaries, making the island shrink in size laterally.
	}\label{tau8}
\end{figure}
Taken together we show that tau on microtubules can co-exist in two kinetically distinct phases, manifested in the reversible formation of cohesive islands, cooperatively compounded of stationary tau molecules with a low turn-over rate. These islands form in a pool of discretely binding, diffusible tau molecules with a high turn-over rate (Figure 7). At physiological concentrations the islands colocalize with a layer of high turn-over tau, which can obscure the position of the islands. Tau islands reversibly regulate the local accessibility of the microtubule surface, and thus locally control katanin severing and kinesin-1 mediated transport. The tau islands described in our study are fundamentally distinct from the patches of clustered tau reported earlier9, as islands form reversibly by growing from their boundaries, forming a uniform layer and shielding the entire accessible surface of microtubules.
Tau islands display a characteristic density (0.25 - 0.45 nm-1), which is similar to the value of complete coverage, estimated recently by cryo-electron microscopy (~0.43 nm-1), suggesting that islands are monolayers of tau as shown in the respective study21. Further increasing the concentration of tau in solution in our experiments resulted in an increase in the density of tau that rapidly turned over on the microtubule surface, obscuring the islands at physiological tau concentrations. A reason why this pool of tau was not captured by cryo-electron microscopy experiments could be the transiency of the interactions, which might be mediated by the C-termini of tubulin as suggested earlier4. Importantly, this phase of rapidly turning over tau molecules was not able to protect microtubules from katanin severing even at saturating densities at micromolar concentrations. By contrast, islands protected the microtubule from katanin at tau concentrations as low as 20 nM, showing that it is not the density of tau molecules, but the cohesion between the cooperatively binding tau molecules constituting the island, which protects the microtubules from degradation by katanin. 
Islands assemble and disassemble predominantly at their boundaries suggesting that the integrity of the islands depends on cooperativity between the constituting tau molecules. Tau is known to undergo liquid-liquid phase separation in solution which is underpinned by tau-tau interactions25. As tau-tau interaction on the microtubule have been reported earlier26, it seems plausible that tau-tau interactions also underpin the formation of islands. Alternatively, or additionally, cooperativity could depend on the local modification of the microtubule surface by tau binding, which translates along the microtubule lattice into adjacent binding sites, increasing the affinity of incoming tau molecules for these adjacent binding sites.	
How can kinesin-8 displace such cohesive tau islands? Tau molecules possess four microtubule-binding-repeats, which separately interact with one tubulin dimer each, occupying similar binding site as the kinesin motor domain21. While tau proteins are bound multivalently to the periodic surface of the microtubule, the four microtubule-binding-repeats of a tau molecule each bind and unbind individually. Importantly, kinesin-8 motors do not dissociate when the next binding site on the microtubule is not available. We hypothesize that this ability enables them to wait until, stochastically, the next binding site becomes available by transient tau repeat unbinding and, in this way, sequentially replace the four microtubule-binding-repeats. During this process the kinesins have to overcome just the affinity of a single microtubule-binding repeat, not the avidity of the whole tau molecule.
The ability of kinesin-8 motors to disassemble the islands suggests that the accessibility of the microtubule surface is governed by an interplay between the avidity of cooperatively binding tau molecules constituting the cohesive islands and affinity of other axonal MAPs. The human kinesin-8, Kif18A, is involved in regulating axonal microtubule dynamics17 and, like its S. cerevisiae orthologue Kip3, employed in the current study, was described in vitro as super-processive molecular motor19,27. Kif18A thus possesses the crucial property required for a molecular motor to displace tau islands and therefore might be involved in regulating the localization of tau in neurons, and reciprocally, might be regulated by tau islands, which, we showed, initiate traffic jam formation at their boundaries, and reduce the speed of the motors that enter the island-covered region. 
In summary, we show that tau on microtubules separates into the kinetically distinct phases over a broad range of conditions. Complementary work presented in Tan et al. confirms the existence of tau phase-separation on the microtubule surface and shows its significance for the regulation of cytoplasmic dynein and spastin, extending our results28. We hypothesize that the islands of stationary tau can tag specific regions on the microtubules, for example as a readout of differential post-translational modifications of tubulin, rendering the regions on the microtubule surface differentially accessible to other MAPs. Sorting of proteins associated with cytoskeletal filaments can be driven by reciprocal exclusion generated e.g. by geometrical constraints29 or localized binding of different intrinsically disordered proteins30. As tau is not the only disordered protein to have a propensity to undergo liquid-liquid phase separation, we hypothesize that other disordered proteins might be also be able to phase-separate on the microtubule surface, which could add another layer of MAP sorting and regulation on microtubules. It is intriguing to speculate that in neurodegenerative diseases, which involve the gradual dissociation of unstructured proteins from neuronal microtubules, diminished island assembly, triggered e.g. by hyperphosphorylation of tau, could be a cause for various downstream (patho-)physiological effects. 

\section{Ase1}
In this study, we examined how the diffusible MT crosslinker Ase1 affects MT depolymerization, both when connecting MTs and on individual MTs. Our findings indicate that Ase1 reduces MT depolymerizing speeds and selectively enhances rescue frequencies of MTs in antiparallel configurations, thereby stabilizing antiparallel overlaps while having minimal impact on parallel overlaps or isolated MTs. In bipolar MT arrays, like mitotic spindles, this attribute of Ase1 could enable a selective stabilization of the array's central regions, while keeping the rest of the array dynamic and pliable. An earlier study on a plant Ase1 analogue, MAP65-1, found increased rescue rates of MTs within bundles compared to isolated MTs \parencite{Stoppin-Mellet2013}. This study, however, did not experimentally distinguish between parallel and antiparallel bundles. Our methods allowed us to directly distinguish between different bundling orientations, and our findings in fact support the modeling-based findings by \parencite{Stoppin-Mellet2013}. However, our results do not rule out that under different experimental conditions, Ase1 may also stabilize parallel bundles to some degree.

We observed Ase1 sweeping, the accumulation of lattice-bound, diffusible Ase1 at the retracting ends of depolymerizing MTs, and found that accumulated Ase1 can transduce forces to other MTs. This phenomenon, also known as "protein sweeping" or "herding," is analogous to the in vitro behavior of the MT-severing enzyme spastin and the kinetochore-associated Ndc80 and Dam1 complexes, which crosslink chromosomes to depolymerizing MT ends \parencite{Franck2007, umbreit2012ndc80, grishchuk2017biophysics}, indicate that, analogously to Dam1- and Ndc80- complexes, Ase1 accumulation at depolymerizing MT ends, antagonizes the dissociation of tubulin subunits from these MT ends. While on isolated MTs this effect may be neutralized, either by Ase1 dissociation or translocation along the MT lattice, both of these processes happen less readily within overlaps, where Ase1 diffusion constant and unbinding rates are greatly reduced \parencite{Kapitein2008, lanskydiffusible2015}, due to protein avidity resulting from the multivalent interactions of Ase1 with the MTs \parencite{braun2020cytoskeletal, erlendsson2021binding}. Additionally, in overlaps, individual protofilaments of a depolymerizing MT are coupled to the other MT by the Ase1 crosslinkers and in order to bend into the rams horn formations associated with depolymerization, likely have to work against the lattice of the other MT, which might lead to further stabilization. This effect is analogous to Dam1- and Ndc80- complexes whose rescue-promoting propensity can be enhanced by exerting load on the complexes \parencite{Franck2007, volkov2018multivalency}. Importantly, this scenario applies only to antiparallel MT overlaps, as for parallel overlaps no stabilizing effect additional to the one observed on isolated MTs arises, indicating that there the force-coupling is weak. This is consistent with the fact that Ase1 binds with lower affinity to parallel, compared to antiparallel MTs. Nevertheless, the mere presence of Ase1 on the lattice of isolated MTs is sufficient to reduce depolymerization velocities. Our modeling suggests that Ase1 may stabilize adjacent protofilaments, which might be explained by Ase1 binding to a tubulin dimer allosterically stabilizing the adjacent protofilament either by Ase1's intrinsically disordered N-terminus directly binding to the adjacent protofilament \parencite{Subramanian2010} or indirectly through the tubulin lattice, such as kinesin-1 \parencite{Peet2018, morikawa2015x}. 

Our modelling further suggests that the ability of Ase1 to both diffuse and reduce tubulin subunit detachment at depolymerizing plus ends confers interesting biological properties. Not only can Ase1 reduce the depolymerization speed of MTs, but this makes Ase1 capable of tracking depolymerizing ends, since subunits without Ase1 are more likely to be lost. Our findings thus suggest that any diffusing molecule that prevents tubulin unbinding will track depolymerizing ends, and therefore may exert forces on objects that the molecule has affinity for on accessible regions as the MT shrinks. Conversely, these forces will drag the molecule in the opposite direction of MT depolymerization, making it more likely to be at the terminal subunits, amplifying its braking effect on depolymerization speed. It may be interesting to measure the forces which Ase1 sweeping is capable of transducing, which would also shed light on the question on whether protofilament powerstrokes are an important component of Ase1 sweeping. Interestingly, starPEG-(KA7)4, a synthetic MT crosslinker with multivalent MT-binding interfaces has recently been shown to also drag MTs when being swept by depolymerizing MTs \parencite{Drechsler2019}, even though it did not hinder MT depolymerization of isolated MTs. It would be interesting to scrutinize the dynamics of bundles crosslinked by starPEG-(KA7)4 in future studies, as well as the depolymerization speed of "pulling" MTs. 

We produced simple models based on the assumption that Ase1 reduces the detachment of terminal tubulin subunits when bound at the MT tip. This assumption, when allowing for diffusion of Ase1 molecules along the protofilament, leads to both a decrease of MT depolymerization velocity and accumulation of Ase1 at the tip of shrinking MTs. Our model quantitatively recapitulated the behavior of the system for 6 nM Ase1, and within an order of magnitude for 1 nM Ase1. Given the low density of Ase1 molecules at 1 nM concentration (<1\% of tubulin dimers bound to Ase1), the discrepancy may be due to stochasticity of the system. There was one more potential discrepancy between our modeling results and our experimental results for isolated MTs, namely that the model predicted a characteristic length $\lambda$ of the exponential decay of Ase1 at the microtubule end of around 600nm, while we measured only around 200nm. However, this signal is hardly comparable with our isolated-protofilament model, not only because it comes from multiple protofilaments that may not be in register, but also because shrinking protofilaments are likely curved outwards \parencite{McIntosh2008}. What could explain the quantitative disagreement with experimental data for antiparallelly crosslinked MTs? In principle, the fact that a disagreement exists is not surprising, given that overlaps are not symmetric, and some protofilaments have almost no Ase1, while others have extremely high Ase1 density (76\% tubulin dimers bound to Ase1 in the MT body assuming 2 protofilaments crosslinked, 50\% assuming 3, see “Mathematical modelling” in Methods). It is for instance conceivable that the crosslinked protofilements lag behind the non-crosslinked protofilements, forming a "tail" (TODO: cite paper that shows such a lagging tail). Hence, a more complex model accounting for protofilament interactions would be needed for overlaps. Such a model would likely need to be informed by experimental measurements of such interactions. However, it is also possible that we simply are overestimating the number of Ase1 molecules herded by antiparallel MTs, because part of the Ase1 which is lost at the depolymerizing MT end presumably remains bound to the template MT, an effect which we do not account for in our estimation. A likely expression of this effect is our observation that while at the ends of isolated MTs, we observed a (blurred) right-sided exponential decay of additional Ase1 density as predicted by our model, the additional Ase1 density at the ends of antiparallel MTs were more reminiscent of a gaussian \pref{ase2e}{B-D}. In particular, the exponential fits often did not fully capture the additional Ase1 density we observed which "lingered" behind the depolymerizing MT end, an additional density which the gaussian fits did capture (notably, because this additional density is likely due to Ase1 molecules still bound to the template MT after detaching from the depolymerizing MT end, we decided to base our analysis as shown in \aref{ase2d}{} on the results stemming from the exponential fits).
Our model did not include MT rescues; however, if one assumes that each crosslink reached by a depolymerizing MT tip has a chance of inducing rescue, as proposed by \cite{Stoppin-Mellet2013}, we expect a positive correlation between Ase1 density and rescue frequency, consistent with our experimental data. Indeed, the relationship we observed appeared to be a linear one, which would precisely be the relationship proposed by \cite{Stoppin-Mellet2013}.\par 

Why did we not observe Ase1 to promote rescues on isolated MTs? It appears likely that this stems from the conformational constraints introduced by the microtubule the depolymerizing microtubule is crosslinked to. Depolymerizing involves a bending of the tubulin subunits at the microtubule tip. However, precisely this bending might be opposed by the other microtubule, as an outward-bending protofilament has to push against it. Under the absence of Ase1 in solution, this is not an issue (as we had shown in \aref{ase1c}{}), because the other microtubule can easily move away from the tip of the depolymerizing microtubule. However, it seems likely that such a movement would require more energy in the case of a crosslinked microtubule. Not only does it appear likely that the terminal Ase1 would oppose such a separation, but also all the other crosslinking Ase1 molecules in the vicinity of the tip. Another potential mechanism increasing the stabilizing effect of Ase1 in the case of antiparallel overlaps could be the multimerization of Ase1 within antiparallel overlaps as reported by \cite{Kapitein2008}, a feature recently shown to play a crucial role in slowing motor-driven MT sliding \parencite{alfieri2021two}. This multimerization, particularly when enhanced by Ase1 herding at depolymerizing MT ends, could introduce additional complexity to Ase1-mediated MT dynamics regulation. The possibility of Ase1 molecules acting cooperatively to promote rescues for antiparallel MTs specifically is intriguing and would offer the cell a lever for modulating the rescue-promoting effect of MT crosslinking. \par
 
Our results show that the presence of diffusible MT crosslinkers can suffice to establish enduring antiparallel MT overlaps, such as those found in the midzone of mitotic spindles. In such context Ase1 can work cooperatively with other MT rescue factors such as CLASP \parencite{Bratman2007b} or provide an alternative mechanism for selective life time enhancement of antiparallel overlaps. We speculate that the impact of diffusible crosslinkers on MT dynamics may be tunable by posttranslational modifications of either the crosslinkers or the MT surface. Such a tunability has recently been proposed for a seemingly related capacity of Ase1, namely the braking of MT sliding caused by molecular motors \parencite{fu2009phospho, Thomas2020}. For actin filament overlaps, it has been observed that F-actin crosslinkers slow down actin depolymerization \parencite{maul2003eplin,schmoller2011slow}, suggesting that crosslinker dependent stabilization of filaments may be a fundamental mechanism, widespread across cytoskeletal systems.


\section{The patterning of cytoskeletal polymers}
Can we draw any insights which go beyond the pecuilarities of each of the two MAPs I was focusing on? It has already been mentioned
- If tubulin bonds were strong laterally, then the cell would have to break up two strong bonds in order to initiate dissassembly -> Maybe would be too complicated, which is why such stabilization does not really take place a lot?
- On the importance of interplay and cooperation?

. Valiron O, Arnal I, Caudron N, Job D: GDP-tubulin incorporation
into growing microtubules modulates polymer stability. J Biol
Chem 2010, 285:17507-17513.
13. Bowne-Anderson H, Zanic M, Kauer M, Howard J: Microtubule
dynamic instability: a new model with coupled GTP hydrolysis
and multistep catastrophe. Bioessays 2013, 35:452-461.
14. Piedra F-A, Kim T, Garza ES, Geyer EA, Burns A, Ye X, Rice LM:
GDP-to-GTP exchange on the microtubule end can contribute
to the frequency of catastrophe. Mol Biol Cell 2016, 27:3515-
3525.