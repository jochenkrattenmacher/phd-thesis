\section{Crosslinking by Ase1 in Vitro Stabilizes Dynamic Microtubules}
Spatial regulation of microtubule (MT) organization and dynamics is critical for the assembly of MT-based structures such as the mitotic spindle (Karsenti et al., 2006; Nedelec et al., 2003). The stabilization of MTs underlies mechanisms of spindle assembly, chromosome segregation, cytokinesis, and polarization of interphase arrays in many cell types. 


 Of the two ends of the microtubule, \textit{in vitro} the plus end polymerizes more quickly and is generally more dynamic \parencite{Howard2003}. \textit{In vivo}, minus ends never polymerize \parencite{dammer}. %For those two reasons, this thesis focuses almost exclusively on the plus end. It has become clear that the plus end intrinsically changes its direction of growth rapidly over time without the need for additional proteins. However, cells can tune the characteristics of the process by employing dedicated MAPs.
 \FloatBarrier
 
 Tubulin is an eukaryotic protein highly conserved throughout evolution, which renders organisms such as fission yeast as viable as models for animal cells in terms of microtubule and MAP behavior.