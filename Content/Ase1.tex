\section{Crosslinking by Ase1 in Vitro Stabilizes Dynamic Microtubules}
\label{sec:Ase1}
\subsection{Motivation}
The spatial regulation of microtubule dynamics is critical for the assembly of microtubule-based structures such as the mitotic spindle \parencite{NEDELEC2003, Karsenti2008}. Eukaryotic cells employ a variety of mechanisms and microtubule-associated proteins to achieve the required degree of regulation.  \par Regarding our specific methodology, it is worth noting that tubulin is an eukaryotic protein highly conserved throughout evolution, which renders organisms such as fission yeast as viable as models for animal cells in terms of microtubule and MAP behavior. Moreover, of the two ends of the microtubule, \textit{in vitro} the plus end polymerizes more quickly and is generally more dynamic \parencite{Howard2003}. \textit{In vivo}, minus ends never polymerize \parencite{dammer}. For these reasons, we exclusively focused on the plus end.
 
 