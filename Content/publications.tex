\chapter{Publications related to this thesis}
\label{chap:publications}
Two publications are related to this thesis, according to which the results part of this thesis is structured. Unless indicated otherwise, all figure panels have been reproduced from the respective publications.

\section{Publication 1: Kinetically distinct phases of tau on microtubules regulate kinesin motors and severing enzymes}
In the first publication \parencite{Siahaan2019a}, titled "Kinetically distinct phases of tau on microtubules regulate kinesin motors and severing enzymes," we share our discovery that there exist two distinct modes of tau-microtubule interaction which result in distinct phases of tau on microtubules. We furthermore show that these phases differ in their interaction with microtubule-associated motor proteins and the microtubule severing enzyme katanin. This research thus has contributed to our understanding of how tau may perform its manifold different regulatory functions (see \autoref{sec:tau}; \cite{Morris2011b} in particular have emphasized this notable versatility of tau).
The results of this work are presented in \autoref{sec:tau} and 
TODO explanation and author contributions.
This publication can be found in appendix A.

\section{Publication 2: Ase1 selectively increases the lifetime of antiparallel microtubule overlaps}
The second publication \parencite{Krattenmacher2024}, "Ase1 selectively increases the lifetime of antiparallel microtubule overlaps," adds to our understanding of how Ase1 affects microtubule dynamic instability (see \autoref{sec:instability}). In particular, our research shows that Ase1, without the help of any other MAP, can give rise to long-lasting antiparallel microtubule overlaps, which are structurally critical features of mitotic spindles. Notably, earlier in-vitro experiments had already shown that MAP65-1 (a Ase1 analogue in plants), when bundling microtubules, promotes microtubule rescues \parencite{Stoppin-Mellet2013}. However, it was still unclear whether, and in how far, microtubules bundled in parallel fashion are affected differently than microtubules bundled in antiparallel fashion (\autoref{sec:MTarrays}). We for the first time also observed Ase1 to directly have an impact on the depolymerization of single microtubules. 


The results of this work are presented in \autoref{sec:Ase1} and have been published in Current Biology \parencite{Krattenmacher2024}. Unless indicated otherwise, the experiments and the analysis have been conducted exclusively by me. The most notable contribution by a third party was by Manuel Lera Ramirez, who had contributed 85 \% of the model-related work (most of the remaining model-related work was performed by me, i.e., work as indicated in the text). A reprint of this publication can be found in appendix B.