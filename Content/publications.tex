\chapter{Publications related to this thesis}
\label{chap:publications}
Two publications are related to this thesis, according to which the results part of this thesis is structured. Unless indicated otherwise, all figure panels have been reproduced from the respective publications.

\section{Publication 1: Kinetically distinct phases of tau on microtubules regulate kinesin motors and severing enzymes}
In this publication in Nature Cell Biology \parencite{Siahaan2019a} (Appendix A), we share our discovery that there exist two distinct modes of tau-microtubule interaction which result in distinct phases of tau on microtubules. We furthermore show that these phases differ in their interaction with microtubule-associated motor proteins and the microtubule severing enzyme katanin. This research thus has contributed to our understanding of how tau may perform its manifold different regulatory functions (see \autoref{sec:tau}; \cite{Morris2011b} in particular have emphasized the astonishing versatility of tau). \par

As mentioned in the declaration at the beginning of this thesis, many of the tau-related experiments were performed by Valerie Siahaan (a member of the Laboratory of Structural Proteins), with whom I shared first-authorship. To be more precise: Both Valerie and I conducted experiments which only feature tau without any other MAPs, as well as experiments which feature tau and katanin. Notably, Valerie conducted most of the experiments related to tau only, while my focus was on the analysis of the data which Valerie had generated. In the case of the experiments with Kinesin-8, experiments were conducted by me, while in the case of the experiments with Kinesin-1, the experiments were conducted and analyzed exclusively by Valerie (and are thus not presented in any figure in this thesis). \par

In this thesis, the results of this work, as well as some additional tau-related data which was not presented in the mentioned publication or any other (indicated as such in the respective figures), are presented in \autoref{sec:tau}. 


\section{Publication 2: Ase1 selectively increases the lifetime of antiparallel microtubule overlaps}
With this publication in Current Biology \parencite{Krattenmacher2024} (Appendix B), we add to our understanding of how Ase1 and potentially other diffusive microtubule crosslinkers affect microtubule dynamic instability (see \autoref{sec:instability}). In particular, our research shows that Ase1, without the help of any other MAP, can, at least \textit{in vitro}, give rise to long-lasting antiparallel microtubule overlaps, which are structurally critical features of mitotic spindles. Notably, earlier in-vitro experiments had already shown that MAP65-1 (a Ase1 analogue in plants), when bundling microtubules, promotes microtubule rescues \parencite{Stoppin-Mellet2013}. However, it was still unclear whether, and in how far, microtubules bundled in parallel fashion are affected differently than microtubules bundled in antiparallel fashion (\autoref{sec:Ase1_intro}), a question which our publication provides answers for. We for the first time also observed Ase1 to directly have an impact on the depolymerization of single microtubules by reducing the speed of depolymerization of microtubules. Via mathematical modeling, we in our article show that this effect may well be related to another phenomenon we report, namely the herding of Ase1 by depolymerizing microtubule ends. \par

As mentioned in the declaration at the beginning of this thesis, most of the experimental work and data analysis related to this publication was done by me. The major exception to this is the experimental work done by Alexandre Beber (as required by the Current Biology reviewers) after I had already left the laboratory, which I indicate in the respective figures. As a more minor point, it is also worth noting that our collaborator Manuel Lera Ramirez provided an intial algorithm for fitting Ase1 densities at the ends of depolymerizing microtubules (which I then adjusted). Also, again as noted in the declaration, Manuel did most of the modeling-related work. To be more accurate, I would estimate that he had contributed approximately 85 \% of the modeling-related work (most of the remaining model-related work was performed by me, i.e., work as indicated in the text).\par

In this thesis, the results of this work are presented in \autoref{sec:Ase1}.