\chapter{Methods}
\label{methods}

\section{Microtubule preparation}
\subsection{Tubulin preparation}
Tubulin was extracted from pig brains and labeled following previously described methods \parencite{CASTOLDI200383}. Fresh pig brains were cleaned and homogenized in a blender using an ice-cold depolymerization buffer, then centrifuged at 29,000 x g for 60 minutes at 4 ºC. The supernatant was then mixed with an equal volume of high-molarity PIPES, supplemented with 1.5 mM ATP and 0.5 mM GTP, and warmed glycerol at 37 ºC to induce microtubule polymerization. This mixture was incubated for 60 minutes at 37 ºC, followed by centrifugation at 150,000 x g for 30 minutes at 37 ºC. The resulting microtubule pellet was depolymerized by resuspension in ice-cold depolymerization buffer, dounced on ice for 10 minutes, and then incubated for an additional 60 minutes on ice before centrifuging at 70,000 x g for 30 minutes at 4 ºC. Subsequently, the supernatant containing depolymerized tubulin was diluted with equal volumes of prewarmed high-molarity PIPES and glycerol, supplemented with 1.5 mM ATP and 0.5 mM GTP, incubated at 37 ºC for 30 minutes to promote microtubule polymerization, and centrifuged again at 150,000 x g for 30 minutes at 37 ºC. Finally, the microtubule pellet was depolymerized by resuspension in ice-cold BRB80 and dounced on ice for 10 minutes. After an additional 10-minute incubation on ice, the solution was centrifuged at 100,000 x g for 30 minutes at 4 ºC (SW 41Ti rotor). Tubulin was aliquoted, flash-frozen in liquid nitrogen, and stored at -80 ºC for later use. Some tubulin was aliquoted at high concentrations for subsequent labeling (performed by a laboratory assistant), as described by \cite{HYMAN1991478}. Briefly, tubulin labeling involved polymerizing microtubules, incubating them with fluorescent dye, and then depolymerizing the microtubules to obtain labeled tubulin.

\subsection{Microtubule polymerization}
For preparation of biotinylated microtubules, isolated tubulin was mixed with biotinylated tubulin (Cytoskeleton Inc., T333P) at 50:1 mass ratio. For preparation of labeled microtubules, isolated tubulin was mixed with labeled tubulin. We employed two microtubule stabilization techniques \pref{sec:stabilization}{}: i) Polymerizing microtubules under the presence of GMPCPP. ii) Stabilization of microtubules by having paclitaxel present in the buffer \parencite{SCHIFF1979}. GMPCPP-stabilized microtubules were grown using a mixture of 2 $\mu$M tubulin, 1 mM GMPCPP and 4 mM MgCl$_2$ in BRB80 and incubated for 3 hours at 37°C. Paclitaxel-stabilized microtubules were polymerized using a mixture of 1 mM GTP, 4 mM MgCl$_2$ and 5 \% DMSO in BRB80 for 30 minutes and subsequently stabilized by diluting the mixture in BRB80 + 10 µM paclitaxel. In both the GMPCPP and the paclitaxel procedure, the resulting mixture was then spun at 12000 g in a tabletop centrifuge. Finally, the supernatant was discarded, and the pellet was resuspended in 50 $\mu$l BRB80, respectively BRB80 + 10 µM paclitaxel. The paclitaxel-stabilized microtubules were not used for assays with dynamic microtubules, as the paclitaxel would cause uninterrupted growth and inhibit catastrophes.

\section{Sample Preparation}
\label{assayPREP}
To conduct our microscopy, we implemented a procedure which has already been described earlier \parencite{Gell2010a}. We manufactured microfluidic channels as depicted in \autoref{Gell2010a_setup}A and after some further preparation steps as described below mounted these channels on the microscope, e.g., for TIRF microscopy as shown in \autoref{Gell2010a_setup}B. The coverslips used in the process, after a cleaning procedure, were functionalized with dichlorodimethylsilane (DDS) to allow for antibody binding to the surface.\par
\begin{figure}[htb]
\centering
\includegraphics[scale=1.1]{Figures/setup.png}
\caption[The flow cell layout and handling, adapted from \parencite{Gell2010a}.]{
		\textbf{A sketch of the flow cell layout and handling, adapted from \parencite{Gell2010a}}. The four depicted Parafilm stripes were put as spacers in between two DDS-coated glass slides to form three channels. To seal the channels, this construct was then heated up for about 30 seconds while gently pressing the upper slide onto the lower. Next, it was clamped into a brass sample holder. To fill the initially dry channels with liquid, vacuum was employed. Further perfusion steps were conducted by simply utilizing a filter paper as is illustrated. \textbf{B} Schematic of our our experimental setup. The labeling of our microtubules and the included assay components varied from experiment to experiment. 
	}\label{Gell2010a_setup}
\end{figure}
After manufacturing, flow channels were incubated with antibodies in PBS for 1 to 5 minutes (to immobilize biotinylated microtubules, we used anti-biotin antibodies, to immobilize microtubules without biotin, we used anti-$\beta$-tubulin antibodies), followed by incubation for at least 30 min with 1\% pluronic F-127 in PBS to prevent unspecific protein binding. The flow channel was then washed with BRB80 prior to addition of microtubules for antibodyspecific binding ("template microtubules"). Unbound microtubules were then removed in another wash step.\par
After these preparatory steps (and in some cases additional steps, see below sections), the assay buffer was added, the flow chambers were sealed in the case of longer experiments to prevent changes in concentrations due to evaporation, and the coverslip holder was mounted onto the microscope stage (setup shown in \autoref{Gell2010a_setup}).

\section{Imaging}
Atto647-labeled microtubules, mCherry- and mEGFP-labeled proteins were visualized sequentially by switching between the Cy5, TRITC and GFP channels (Chroma filter-cubes) using Nikon-Ti E microscope equipped with 100x Nikon TIRF objective and either Hamamatsu Orca Flash 4.0 sCMOS or Andor iXon EMCCD cameras. In the case of tau experiments, the acquisition rate varied between 1 frame per 30 ms to 1 frame per 30 seconds depending on the particular experiment and is indicated in the corresponding figure. In the case of one set of Ase1 experiments (Set A, see \autoref{Ase1AssayMethods}), the GFP channel was visualized or the IRM channel (or both with sequential switching), at a framerate of 5s. For the other set of Ase1 experiments (Set B), channels were sequentially switched at a framerate of 2.6 seconds (with a 633x Zeiss oil immersion TIRF objective in combination with a Andor iXon DV 897 (Andor Technology) EMCCD camera). Imaging conditions in experiments used for quantitative estimation of kinetic parameters were set such that photo-bleaching effects were negligible (< 2 \% fluorescent intensity loss during the experiment). Finally, for Ase1 Set A experiments, the Alexa647-labeled microtubule seeds were imaged before the start of the time lapse, and only the Ase1-mNeonGreen channel was imaged during the time lapse. For Set B experiments, the rhodamine (tubulin) and the GFP (Ase1-GFP) channel where imaged sequentially, whereas every 40th frame the Alexa647 channel was imaged in place of the GFP channel, in order to track the location of the GMPCPP-stabilized seeds (which we with this data determined to not move significantly during experiment time). 

\section{Image analysis}
\label{methods_analysis}
Data was analyzed using FIJI \parencite{Schindelin2012} and custom written Matlab (Mathworks) routines. 
\subsection{Density estimation}
Kymographs (KymographBuilder plugin, custom-modified to compute integrated intensity instead of finding the maximum intensity, see Zenodo: \url{doi.org/10.5281/zenodo.3270572}) along the microtubule length were used to read out the fluorescent signal and to estimate the integrated signal intensity of fluorescent proteins bound to the microtubule (if necessary, time series were drift-corrected with FIESTA \parencite{RUHNOW20112820}). The recorded signal in regions directly adjacent to the microtubule was subtracted as background signal. Kymograph pixels were then manually categorized according to the type of microtubule region they covered (island, curved microtubule, regions surrounding the islands in case of tau, overlapping microtubules or single microtubules in the case of Ase1). The integrated intensity averaged along the microtubule length for each region type was then computed by taking the mean of the categorized kymograph pixels (in the case the Ase1 assays, only regions with dynamic extensions present were measured). The density of labeled MAPs bound to the microtubule was then estimated by dividing the averaged integrated intensity by estimated intensity per single molecule times unit length. Conversion to the number of MAP molecules per tubulin dimer was performed assuming 13 available protofilaments and 8 nm length of a tubulin dimer.

\subsection{Diffusion coefficient estimation} 
Single tau molecule tracking for the estimation of diffusion coefficients was performed using FIESTA \parencite{RUHNOW20112820} software. For reconnecting tracks, a threshold velocity of 12000 nm/s had been chosen, and tracks were allowed to have at most 3 missing frames between two data points. To minimize false-positive connecting of separate molecules, the tracks obtained by FIESTA were cut into pieces such that the maximum distance between two data points was never above 360 nm.

\subsection{Fluorescent signal of a single fluorescent molecule} 
In our assays, we often were interested in the absolute number of labeled proteins bound to microtubules. To estimate this number, it was necessary to know the contribution of single fluorophores to the measured signal. The fluorescent signal of a single fluorescent molecule was determined by generating intensity time-traces of single fluorophore-labeled kinesin-1 molecules tightly bound to the microtubule in presence of AMP-PNP (in the absence of ATP) and estimating the height of the occurring bleaching steps. The number of steps was first estimated by eye, and this number was used as input for the \textit{findchangepoints} function of Matlab to determine the position of the steps (see \autoref{bleaching_steps}). To yield the intensity per single molecule, the heights of these steps were averaged (and in the case of Ase1, multiplied by two, given that Ase1 forms homodimers).
\begin{figure}[htb]
\centering
\includegraphics[scale=1.1]{Figures/bleaching_steps.png}
\caption[An illustration explaining our estimation of the fluorescent signal of a single fluorophore.]{\textbf{An illustration explaining our estimation of the fluorescent signal of a single fluorophore.}
		The signal recorded on a particular spot on a microtubule decorated with immobilized fluorophore-labeled kinesin-1 molecules. The signal shows a step-wise decay due to photobleaching of the fluorophores. The bleaching steps were detected detection of significant changes of the mean values.
	}\label{bleaching_steps}
\end{figure}

\section{Data representation}
In all boxplots presented in the figures, horizontal midline indicates the median; bottom and top box edges indicate the 25th and 75th percentiles, respectively; the whiskers extend to the most extreme data points not considered as outliers (the function Alternative box plot from the IoSR Matlab Toolbox has been used); the numbers indicate the sample size; the notches are centered on the median and extend to ±1.58*IQR/sqrt(sample size). Where single, colored data points are presented, points from the same experiment are indicated by the same color (unless otherwise stated). 

\section{Procedures specific to tau experiments}
\subsection{Protein expression and purification} 
mEGFP or mCherry tagged tau and tau$\Delta$N, Kinesin-1, Kip3 and katanin (M. musculus katanin p60/p80C, \cite{Jiang2017}) were expressed and purified as described previously \parencite{HERNANDEZVEGA20172304,Herrmann2018,Mitra2018,NITZSCHE2010247}. In particular, all tau variants \pref{tauconstructs}{} were purified from insect cells using the baculovirus expression system. Recombinant baculovirus for each construct was produced as described by \cite{woodruff2015method}. Sf9 insect cells in log phase (1 million cells/ml, Expression system) were infected with 5 ml of P2 baculovirus stock, incubated at 27°C with moderate shaking, and harvested 72 hours post-infection. The cells were collected by centrifugation at 700g for 8 minutes and then resuspended in resuspension buffer [25 mM HEPES, 150 mM KCl, 20 mM imidazole (pH 7.4) with 1 mM DTT, 1 mM PMSF (Sigma), and 1x Protease Inhibitors Cocktail (Calbiochem, Type III)]. Cell lysis was performed using an Emulsiflex (Emulsiflex-C5, Avestin). The lysate was centrifuged at 35,000 rpm for 45 minutes, and the supernatant was collected. The supernatant was filtered through a 0.45 µm filter and incubated with Ni-NTA agarose resin (QIAGEN) HiTrap for 1 hour. The beads were collected and washed using disposable gravity columns (20 mL, Biorad). The columns were washed four times with 20 ml of wash buffer (25 mM HEPES, 150 mM KCl, 30 mM imidazole, 1 mM DTT, pH 7.4) and eluted with an elution buffer (same buffer containing 250 mM imidazole). The 6xHis tag was removed by treatment with PreScission protease (3C HRV protease, 1:100, 1 µg enzyme/100 µg of protein) overnight at 4°C. Imidazole was removed by dialysis (slide-a-lyzer with a 20 KDa cut-off) overnight at 4°C, with His tag cleavage occurring simultaneously with dialysis. The protein was further purified by size-exclusion chromatography using a HiLoad 16/60 Superdex 200 column with an ÄKTA Pure Chromatography system (GE Healthcare) in 25 mM HEPES, 150 mM KCl, 1 mM DTT (pH 7.4). Collected peak fractions were concentrated to 100 µM or 200 µM using Amicon Ultra 30K (Millipore). Protein concentration was measured using a NanoDrop ND-1000 spectrophotometer (Thermo Scientific) at 280 nm absorbance. Proteins were flash-frozen in liquid nitrogen and stored at -80°C. All steps in the purification were performed at 4°C. 

\begin{figure}[h]
	\centering
	\includegraphics[width=0.6\linewidth]{Figures/tauconstructs.png}
	\caption[Schematics showing the tau constructs used in this study.]{
		\textbf{Schematics showing the tau constructs used in this study.}
		}\label{tauconstructs}
\end{figure}

\subsection{In vitro tau-microtubule binding assay}
Biotinylated, paclitaxel-stabilized, Atto647-labeled microtubules in BRB80T (80 mM Pipes/KOH pH 6.9, 1 mM MgCl2, 1 mM EGTA, 10 µM paclitaxel) were immobilized in a flow chamber using biotin antibodies \pref{assayPREP}{}. Subsequently, assay buffer was flushed into the flow cell (20 mM HEPES pH 7.2, 1 mM EGTA, 75 mM KCl (unless stated otherwise in the main text), 2 mM MgCl2, 1 mM ATP (+Mg), 10 mM dithiothreitol, 0.02 mg/ml casein, 10 µM paclitaxel, 20 mM d-glucose, 0.22 mg/ml glucose oxidase and 20 µg/ml catalase). Then, tau in assay buffer was flushed in at the final assay concentration stated in the main text. In experiments involving multiple successive tau additions, the flow cell was rinsed between each tau addition with a high ionic strength buffer (200 mM KCl in addition to the assay buffer). To remove tau from the solution, the chamber was perfused with approximately four times the chamber volume using assay buffer without tau. For high tau concentrations (>200 nM), larger volumes (up to ten times the chamber volume) were used to ensure complete removal of tau. In experiments with kinesin-8, katanin, or kinesin-1, islands were preformed before introducing the respective protein into the solution, while maintaining a constant tau concentration. In the katanin experiment with elevated tau concentration (see \aref{taukatanin}{}), microtubules were first incubated with 0.8 µM tau-mCherry for 5 minutes. Tau-mCherry was then temporarily removed from the measurement chamber for less than 1 minute to identify the position of the islands, which were obscured by the high tau-mCherry density in the surrounding areas. Tau-mCherry was then reintroduced at 0.8 µM, and after 5 minutes, 215 nM katanin-GFP was added to the solution, while maintaining the tau concentration at 0.8 µM. All experiments were conducted at room temperature.


\subsection{Coverage by tau islands}
The proportion of microtubule length occupied by tau islands was determined by representing both islands and microtubules as segmented lines, measuring their respective lengths, and then calculating the ratio of the total island length on a given microtubule (or within a field of view) to the length of that microtubule (or the combined length of all microtubules in the field of view).

\subsection{Estimation of velocities}
The rates of assembly and disassembly of island boundaries (in the absence or presence of katanin or Kip3) were determined by fitting straight lines to the advancing or retreating edges of tau islands in kymographs. The velocity provided in the text represents a duration-weighted average of these segments. To convert this velocity to the number of tau molecules per second, it was multiplied by the estimated density of tau within the islands (in molecules per nanometer), assuming tau binds to 13 protofilaments with a tubulin dimer length of 8 nm. The velocities of Kip3 and kinesin-1 were determined by fitting straight lines to the kymographs of the moving motors, both inside and outside the islands.

\subsection{Estimation of the tau unbinding time}
To determine the unbinding times of tau inside and outside the islands, the decay in tau density over time was analyzed after a buffer exchange either removing tau from the solution or replacing tau-mEGFP with tau-mCherry. Each analyzed region (island or surrounding area) provided a time trace of tau density decay following the buffer exchange. Time traces from representative experiments were combined, as shown in \aref{tauSHRINK}{E,F}, with the thick line representing the median of all traces at each time point and the bounds showing the first and third quartiles. To estimate the mean residence times of tau inside and outside the islands, individual density time traces were fitted with an exponential decay using the Matlab function fit, excluding data points before the solution exchange. The fits and mean residence times presented were calculated by averaging the coefficients from the individual fits.

\subsection{Katanin severing rate estimation}
Severing rates in regions surrounding the islands were determined by fitting an exponential decay to the number of pixels corresponding to the original microtubule position that exceeded a manually set threshold, which encompassed the microtubule. In island regions, the severing events were counted. In \aref{taukatanin}{B}, the estimated severing rates include both straight and curved microtubules. In \aref{taucurve}{H}, the severing rates are categorized based on the following criteria: straight microtubules were defined as stretches where the orientation did not change by more than 10 degrees, while curved regions were defined as 0.5 µm long stretches centered at the point of greatest curvature with a radius of less than 2.5 µm.

\section{Procedures specific to Ase1 experiments}
\subsection{Protein expression and purification}
Ase1-GFP \parencite{Janson2007} and Ase1-mNeonGreen were expressed in E. coli strain BL21 (DE3) (Altium International). After harvesting the cells, the cell pellet was resuspended in 5 mL ice-cold phosphate buffered saline (PBS) and stored at -80°C for further use. For cell lysis, the cells were homogenized in 30 mL of ice-cold His-Trap buffer (50 mM Na-phosphate buffer, pH 7.5, 5\% glycerol, 300 mM KCl, 1 mM MgCl2, 0.1\% Tween 20, 10 mM BME, 0.1 mM ATP) supplemented with 30 mM imidazole, Protease Inhibitor Cocktail (cOmplete, EDTA free, Roche), and benzonase (Novagen) at a final concentration of 25 units/mL. The mixture was then sonicated and subsequently centrifuged at 45,000 x g for 60 minutes at 4°C using an Avanti J-26S ultracentrifuge (JA-30.50Ti rotor, Beckman Coulter). The clarified cell lysate was incubated with Ni-NTA resin (HisPur Ni-NTA Superflow Agarose, Thermo Scientific) pre-equilibrated with lysis buffer for 2 hours at 4°C. The resin was then sequentially washed with wash buffer I (His-Trap buffer with 60 mM imidazole) and wash buffer II (His-Trap buffer with 60 mM imidazole and 700 mM NaCl). Ase1-GFP was eluted using His-Trap buffer supplemented with 300 mM imidazole. For Ase1-mNeonGreen, after the second wash with buffer II, the resin was further washed with wash buffer I containing 3C PreScission protease (Merck Millipore), which cleaved Ase1-mNeonGreen from the column at the 3C protease cleavage site between mNeonGreen and the 6xHis-tag (the Ase1-mNeonGreen construct had been created by Lenka Grycova). The mixture was incubated overnight at 4°C. The following day, the beads were removed, and the cleaved protein was collected. Both Ase1-GFP and Ase1-mNeonGreen were concentrated by centrifugation at 3500 RPM at 4°C using a 100 kDa centrifugal filter tube (Amicon Ultra-15, Merck). Ase1-mNeonGreen underwent a second purification step through size exclusion chromatography using a Superdex 200 10/300 column. The size exclusion buffer consisted of 100 mM Tris, 150 mM NaCl, 1 mM MgCl2, 1 mM DTT, 0.05\% Tween, 0.1 mM ATP, and 10\% glycerol. Fractions containing the protein were collected and concentrated. The final protein concentrations were measured with a NanoDrop ND-1000 spectrophotometer (Thermo Scientific) at absorbances of 280 and 506 nm. The protein was flash-frozen in liquid nitrogen and stored at -80°C. All purification steps were performed at 4°C.

\subsection{In vitro Ase1-microtubule binding assay}
\label{Ase1AssayMethods}
Biotinylated, GMPCPP-stabilized, fluorescence-labeled microtubules in BRB80 (80 mM Pipes/KOH pH 6.9, 1 mM MgCl2, 1 mM EGTA) were flushed into the prepared channels, were given time to bind to biotin antibodies, and were removed from solution, as described in \autoref{assayPREP}. In the case of Ase1 Set B experiments, additional preparatory steps occured: First, a buffered solution with a low concentration of Ase1 was flushed in, Ase1 was allowed to sparsely bind to the template microtubules, removed from solution, and subsequently non-biotinylated GMPCPP-stabilized microtubules were flushed into the flow cell which got crosslinked to the template microtubules by the Ase1 on these microtubules. These microtubules were labelled with both rhodamine and Alexa647 (represented in sketches in dark blue), while the templates in these assays were only very weakly labelled with Alexa646 (represented in sketches in light blue). Then, the "transport" microtubules were removed from solution so that only transports which formed overlaps with templates would remain in the channel. \par 
The buffer in the flow cell was then exchanged for assay buffer. Finally, Ase1 in assay buffer was flushed into the flow cell at the final assay concentration stated in the main text, together with tubulin, and the channels were sealed. Set A experiments (shown in Figures \ref{ase1a}, \ref{ase1b}, \ref{ase1c}A-E, \ref{ase1d} and \ref{ase2a}) were performed at room temperature and with 32$\mu$M unlabeled tubulin present in solution. Set B experiments (shown in Figures \ref{ase1e}, \ref{ase2b}, \ref{ase2c}C,D and \ref{ase2e}) were performed at 29°C and with 14$\mu$M tubulin, 7\% of which was labeled with rhodamine. The following buffer components common to all used buffers in experiments involving Ase1: 20mM PIPES pH 6.9, 10mM HEPES pH 7.2, 0.5mM EGTA, 1mM MgCl2, 0.5mM Mg-ATP, 0.67mM GTP, 0.67\% Tween20, 6.7mM DTT, 0.3 mg/ml Casein, 13.5mM D-Glucose, 0.3mg/ml glucose oxidase and 0.03mg/ml catalase. The buffer for Set A experiments, in addition to these components, contained 70mM KCl, and 0.1\% Methylcellulose, 0.1\% Glycerol, 1mM sodium phosphate and 1µM ATP. The buffer for Figure Set B experiments, in addition to the components common to all buffers, contained 116mM KCl and 0.065\% Methylcellulose. Experiments shown in Figures \ref{ase1c}F and \ref{ase2c}A,B were performed at the same experimental conditions as Set A experiments (with small differences dependent on the particular experiment as stated in the main text/figure captions).

\subsection{Estimating Overlap Lifetime}
The lifespan of microtubule overlap regions was determined for two distinct microtubule configurations: Antiparallel "midzones," where two dynamic extensions converged to form a dynamic midzone (as depicted in \aref{ase1a}{ left panel}), and parallel bundles composed of two dynamic extensions (as shown in \aref{ase1a}{ right panel}). In both cases, whether antiparallel midzones or parallel bundles, the lifetime was considered to begin when the dynamic (GDP) lattices of each participating microtubule were crosslinked (for antiparallel configurations, an additional condition was that both plus ends had to be within 3 microns of each other at the event's start) and ended when one of the microtubules depolymerized back to its GMPCPP-stabilized segment. For antiparallel bundles, the lifetime also concluded when the midzone was no longer present. If an overlap region persisted until the time-lapse movie concluded, the event was classified as censored. \aref{ase1a}{A} and \aref{ase1c}{B} were created using the Matlab function ecdf with the “survival” option enabled.

\subsection{Adjustment for Ase1 Signal Measurement}
To measure the equilibrium density of Ase1, the signal per unit length (S) detected on isolated microtubules was used to adjust for the reduced illumination intensity in the outer regions of the field of view (when a region of interest (ROI) was located in those regions): $S_{\textrm{corrected}}(ROI) = S(ROI) * S(\textrm{isolated microtubule in center of field of view}) / S(\textrm{isolated microtubule near ROI})$.

\subsection{Determining Microtubule Dynamic Instability Parameters}
The parameters of microtubule dynamics for Set A experiments were estimated by generating kymographs and fitting straight lines to track the position of microtubule plus ends over time and space (using the Ase1-mNeonGreen signal to visually track microtubule ends, as microtubules were not directly imaged). For Set B experiments, the FIESTA software was employed to pinpoint the locations of microtubules \parencite{RUHNOW20112820}. Both methods provided measurements of polymerization and depolymerization velocities. Rescues were identified as instances where a microtubule transitioned from depolymerization to polymerization before reaching the GMPCPP-stabilized seed, while catastrophes were noted when polymerization switched to depolymerization. The frequencies of rescues and catastrophes were calculated by dividing the number of rescues and catastrophes by the total distance depolymerized and polymerized by all plus ends, respectively. In the case of Set A experiments, we tested 10nM during the revision phase, a time when room temperatures were less stable. Consequently, microtubule velocities at all Ase1 concentrations differed from our initial experiments. To allow pooling these results with our initial data, we adjusted the velocities from these experiments by multiplying them by a factor calculated as follows: The mean polymerization and depolymerization velocity of isolated microtubules at 42nM from the initial experiments was divided by the mean respective velocity of isolated microtubules at 42nM from the revision-phase experiments (these mean velocities were weighted by the duration of each polymerization or depolymerization event). The resulting adjustment factors were 0.4 for polymerization and 0.39 for depolymerization.

\subsection{Estimation of amount of Ase1 being swept}
To estimate the number of swept Ase1 molecules for corresponding panels in \aref{ase2b}{} (Set B experiments), we first obtained density traces for each frame during a microtubule depolymerization period. These traces were obtained by summing the pixel intensities perpendicular to the microtubule, i.e., by generating kymograph where each pixel represents such a sum (\url{doi.org/10.5281/zenodo.3270572}). For each frame f we analyzed the corresponding density trace $D_f$ as follows. (1) We computed $D_s$ by subtracting the density trace $D_{before\_catastrophe}$ of the microtubule before the catastrophe had occurred from $D_f$ ($D_s = D_f - D_{before_catastrophe}$) (2) We obtained $x = 0 = X_{Dsmax}$, the location of the local maximum of $D_s$ in vicinity of the microtubule plus end. (3) We obtained $X_{Dsright}$ by finding the first local minimum of $D_s$ to the right of $X_{Dsright}$ (to reduce the effect of noise, we smoothed $D_s$ for this computation). “Right” of $D_s$, in the here-chosen coordinate system, means toward the microtubule seed ($x > 0$). (4) $X_{Dsleft} = X_{Dsmax} -$ 471nm (471 nm = 3 pixels). (5) We computed $D_A$. $D_A$ is equal to $D_f$ to the left of $X_{Dsmax}$, and equal to $D_s + Df(X_{Dsmax}) - D_s(X_{Dsmax})$ to the right of $X_{Dsmax}$. (6) We fitted a distribution $Y_F$ (shape see below) plus an error function $Y_E$ to $D_A$ between XDsleft and $X_{Dsright}$. We required both $Y_F$ and the error function to not have any x-offset: $Y_F$ was a right-sided decaying exponential $exp^{-x/\sigma}$ ($Y_F = 0$ where $x < 0$, and with $\lambda$ bounded between 1 and 1000 nm) convolved with a gaussian $exp^{-x2/2\sigma2}$ (with $\sigma$ bounded between 180 to 190 nm to account for the point spread function of our setup; this same $\sigma$ had been used as input for $Y_E$). Instead of a blurred right-sided decaying exponential, we for some fits (shown in \aref{ase2e}{}) used a gaussian $exp^{-x2/2\sigma_G2}$ for $Y_F$ (with a $\sigma_G$ between 180 nm and 450 nm, which was independent of the $\sigma$ used for blurring $Y_F$). We also fixed $G+E$ (plus a constant value) to approach the minimum of $D_A$ to the left of the end, and the average of $D_A$ to the right of $X_{Dsright}$ (the average of $D_A$ within 5 microns from $X_{Dsright}$, giving more weight to values close to $X_{Dsright}$). (6) We then summed the Ase1 density below $Y_F$ (as discretized in x by the pixel size), which we took as a proxy for the number of swept Ase1-GFP molecules after dividing by the intensity per Ase1 dimer (obtained as described above). 

\subsection{Fluorescence recovery after photobleaching (FRAP)}
Biotinylated GMPCPP-stabilized microtubules were immobilized on the coverslip. We then flushed in the same assay buffer as for Set A experiments, incubated until the Ase1 density on microtubules reached a steady-state, and subsequently bleached Ase1-mNeonGreen molecules and recorded the recovering Ase1-mNeonGreen signal. We fitted the resulting recovery curve to the expression $D_s - c exp(-bt)$, where $D_s$ is the steady state density, and $c$ and $b$ are fitting parameters (see \aref{sec:FRAP}{}). Results for fitting parameter $b$ are shown in Figure S4D.

\subsection{Mathematical modelling}
The scripts to reproduce the modelling, and to plot experimental and theoretical results from \aref{ase2d}{} and \aref{ase2steady}{} can be found in Zenodo: \url{doi.org/10.5281/zenodo.12169420}. 
\subsubsection{Assumptions}
The model of Ase1 accumulation on depolymerizing microtubules, and its effect on depolymerization velocity \pref{ase2d}{A} is built on the following assumptions:
\begin{enumerate}
	\item We neglect interactions between protofilaments and only consider a one-dimensional lattice, where lattice of size $a=8$nm start at index $i=1$ at the plus end, extending to $i=400$. 
	\item Only bound Ase1 molecules are considered by recording the presence or absence (0 or 1) of Ase1 in each lattice site. Bound Ase1 molecules exchange with solution with two constant rates ($k_{on},k_{off}$). Binding is only allowed if the lattice site is empty \pref{ase2d}{A}. $k_{off}$ was directly measured, and $k_{on}$ was adjusted to match the Ase1 equilibrium density on microtubules \pref{ase2t1}{}.
	\item Ase1 particles on the lattice undergo unbiased diffusion characterized by a constant hopping rate ($k_h$). Hopping is only allowed to an empty site \pref{ase2d}{A}. The rate $k_h$ is calculated from the experimentally measured diffusion coefficient of Ase1 \pref{ase2t1}{}, as $k_h=D/a^2$.
	\item The Ase1 particle in the terminal site ($i=1$), cannot hop past the microtubule end (red arrow on the left of \aref{ase2d}{A}), but can detach with rate $k_{off}$.
	\item The terminal lattice site may dissociate from the microtubule, with rate $k_d$ which depends on the presence of Ase1, according to each model:
	\begin{enumerate}
		\item In Model 1, it occurs with rate $k_d^0$ if the terminal lattice site is not occupied \pref{ase2d}{B, top}, and with rate $(1-\Omega)k_d^0$ if it is occupied \pref{ase2d}{B, bottom}. $\Omega$ is a parameter between zero and one. If $\Omega=0$, the presence of Ase1 has no effect, and if $\Omega=1$, the first tubulin subunit cannot unbind if it is bound to Ase1.
		\item In Model 2, the rate of tubulin subunits loss at the plus end is reduced by a factor ($1-\Omega$) if any of the N terminal sites is occupied. At steady state, this rate is $k_d=k_0^d [1-\Omega(1-\prod_{i=1}^{i=N}(1-P_i))]$, where $P_i$ is the probability of site $i$ being occupied by Ase1.
	\end{enumerate} 
	$k_d^0$ is derived from the depolymerization rate of microtubules in the absence of Ase1 ($v_0$), measured experimentally \pref{ase2t1}{}, such that  $k_d^0=v_0/a$.
	\item If the terminal lattice site dissociates when a molecule of Ase1 is bound to it, this Ase1 is lost as well \pref{ase2d}{B, bottom}.
\end{enumerate}
	
\subsubsection{Simplification to a system of constant size}
Since terminal subunits are more likely to be lost when they are without Ase1 than when they are with Ase1, any dissociation event increases the density of Ase1 remaining on the microtubule. This effect is only present at the microtubule end, and away from the end, the probability of a binding site being occupied is only determined by the binding and unbinding constants: $\alpha=k_{on}/(k_{on}+k_{off})$.
Therefore, we can restrict the model to a section of the microtubule with $L$ lattice sites, as long as the probability of finding a molecule at position $L$ is close to $\alpha$. When a depolymerisation event happens, we shift the lattice indexes such that site $i+1$ becomes site $i$, and set $P_{i=L}= \alpha$. 

\subsubsection{Mean field theory}
The system can be solved using a mean-field approximation, by just considering the ensemble of  $P_i$, the average probability of a site i being occupied and neglecting higher-order correlations between neighbouring sites. We can then write a set of discrete differential equations to represent the dynamics of the system:
\begin{equation}
	\frac{dP_i}{dt}=(P_{i+1}+P_{i-1}-2P_i ) k_h+(1-P_i ) k_{on}-P_i k_{off}+(P_{i+1}-P_i ) k_d
\end{equation}

Specific equations apply at the boundaries $i=1$ and $L$:
\begin{align}
	\frac{dP_1}{dt} &= k_h (P_2-P_1 )-P_1 k_{off}+(1-P_1 ) k_{on}+k_d P_2-k_d^0 P_1 (1-\Omega) \\
	dP_L/dt &= 0
\end{align}

The terms of the equation are associated with the rates of diffusion, binding, unbinding ($k_h,k_(on ),k_{off}$) which are constant, and the depolymerization rate ($k_d$), which is affected by lattice occupancy in a different way in each model (see Assumptions).\\
For Model 1, $k_d=k_d^0 (1-\Omega P_1)$.\\
For Model 2, $k_d=k_d^0 [1-\Omega+\Omega\prod_{i=1}^{i=N}(1-P_i)]$.\\
This dynamical system can be evolved from any initial conditions, converging to the unique steady-state solution for a set of given parameters. Assuming that the microtubule is at binding equilibrium when it starts depolymerizing, we initially set $P_i=\alpha$ for all sites. From those initial conditions, we integrate the equations numerically using Python’s \textit{odeint} function (see source code).

\subsubsection{Modelling of overlaps}
To model microtubule overlaps \pref{ase2d}{E,G}, we assume that the Ase1 measured in the overlaps (see Image Analysis above) is evenly distributed among 3 protofilaments that are involved in crosslinking the microtubules. We neglect the other protofilaments. We had also modelled 2 protofilaments instead of 3, which did not fit the experimental data as well as 3 protofilaments.

\subsection{Comparison of experimental data and model}
To compare the predicted and observed timescale of Ase1 accumulation ($\tau$) and Ase1 accumulation at steady state ($A_{end}$), the accumulated Ase1 as a function of time was fitted to $A_{end} (1-exp^{-t/\tau})$ in experiments and model predictions (e.g., \aref{ase2d}{F} for isolated microtubules at 6nM of Ase1). In the model, the accumulation of Ase1 at any given timepoint is defined as $(\sum_{i=0}^{i=L}Pi)-\alpha L$. As a proxy of velocity of depolymerization at steady state, we used the average velocity of depolymerization observed after 20 seconds of depolymerization in experiments and compared it to the depolymerization velocity at the last simulated timepoint \pref{ase2d}{F}. The 95\% confidence intervals of these magnitudes were estimated using the bootstrap method. For each experimental condition with N depolymerization events, a thousand sets of N depolymerization events were drawn through sample with replacement. For each of those sets, $\tau$ and $A_{end}$ were calculated by fitting all observations in the set, and the average velocity after 20 seconds was calculated. Then, the distribution of each magnitude across all sets was used to calculate the 95\% confidence intervals (see source code).
