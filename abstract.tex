\begin{abstract}
However, the underlying molecular mechanism of how tau as an intrinsically disordered protein can fulfill such regulative functions is still unknown. Here we study how the collective behaviour of tau can influence the activity of motors and severing proteins on microtubules. Using in vitro reconstitution assays, we found that on the microtubule surface tau can form cohesive islands in a pool of diffusible tau molecules. We observed that the islands prevented the interaction of katanin and kinesin-1 motors with the microtubule surface. In contrast, super-processive kinesin-8 motors, which, like katanin and kinesin-1 were able to bind to and move in the regions on the microtubule surface occupied by diffusible tau molecules, additionally, could penetrate the islands where they caused island disassembly. Our results show that two kinetically distinct phases of tau, consisting, respectively, of cooperatively and discretely binding molecules, co-exist on microtubules and differentially regulate the interactions of MAPs with the microtubule surface.
\end{abstract}