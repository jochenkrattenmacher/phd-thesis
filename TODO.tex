Hi Jochen, 

% went through the thesis, looks good!

Minor things only, however one very important – it’s crucial to be consistent with Valerie’s thesis. At university they will look at this very closely.

DECLARATION OF THE AUTHOR 

The author of this dissertation thesis hereby declares that the thesis was written independently and all resources as well as co-authors were properly indicated whenever appropriate. All work presented in the results section was done by the author of the thesis in the laboratory of Structural Proteins. Part of this work about tau phosphorylation (manuscript attached in ap- pendix A.3) was used by Adela Karhanova to obtain a master’s degree at the Charles University (http://hdl.handle.net/20.500.11956/181446). The major part of this work has not been used to obtain the same or any other academic degree. 

that’s what Valerie writes. And, in the figure legends, when appropriate: 

Data analyzed by Jochen Krattenmacher (ref). 

Whatever she shows (without specifically indicating it was done by you) you will have to indicate as done by her (in order to not contradict your ‘declaration’). Please have a look at her thesis. She shows very little. Declaring this, in your thesis, as done by her (both experiment and analysis) will be no problem at all. In order to avoid showing data analysed by you (also to keep things concise and comprehensible) she shows very little (really, first have a look, please, before reading further in this file)

Her thesis you can download here: 
https://www.uschovna.cz/zasilka/QK98B3358T2GKHRH-HJG

currently you say this:

“a share of the results presented in this thesis was also presented in the doctoral thesis by Valerie Siahaan, which was difficult to avoid given our tight collaboration during some of the time of my work” 

Or: page 11 “Valerie conducted most of the experiments”

… but, framing it like this is not really an option. You’d have to be specific. For each figure, each panel (that applies only to one or two figures, no worries) you’d need to say who did what. And, for the panels Valerie shows, it is important/crucial/necessary/vital to make this consistent with her statements. They will check this, at university. Mostly you can just state, globally for the whole figures, that the experiments were done by Valerie, analysis by you. Or, for the motors, for example, you don’t say anything, because you did both. Would be also good to mention Lenka (which you do), and Alex, specifically in the panels he contributed to. Same for Manu. You have to be specific. All of this, however, will not be cross-checked as closely as with Valerie’s thesis.

General and also important: you promise to integrate two papers, highlighting that they’d be concerned with a common theme. But then you never come back to this. Either write a final paragraph of discussion, or remove the promise. 
%RE: Yes I just didn't get to this before my trip, the plan was to do this after. I added this section now


please implement the following changes

% • larger fonts and line spacing

% • add figure titles to figure captions – same as in the List of Figures

% • page 1, background, first sentence, what does the ‘2009’ indicate? is this a typo? Did you attend a talk of his, in 2009? best remove.

% • page 1, section 1.1, first sentence, ‘prominent’ not ‘obvious’ and maybe not ‘the most’, but ‘A’.

• also on first page: ‘TODO’ please write something here, could be just one sentence. Throughout the text, there’s several ‘TODO’ places, please find them all and write a sentence. You can keep it short. 

e.g. page 6: TODO
% RE: Yup I left them there as reminders for me to add/fix something.

% • page 11 “Kinesin-8, experiments were conducted by me, while in the case of the experiments with Kinesin-1, the experiments were conducted and analyzed exclusively by Valerie (and are thus not presented in any figure in this thesis). In this thesis, the results of this work, as well as some additional tau-related data which was not” 
% 	No need to mention any of this. You show what you show, and that’s that, nobody will compare this to the paper.  If anything you show was not done by you, you’ll have to indicate this. Especially if it is presented in Valeries thesis (this is super-important, but not much, because she’s not showing much). What you don’t show (e.g. kinesin-8), you don’t need to mention.
% RE: New version: Notably, Valerie conducted the experiments related to tau only, while I analyzed the data which Valerie had generated. In the case of the experiments with Kinesin-8, experiments were also conducted by me.

% • Page 11 ‘for the first time’ – please remove. this is very much shunned upon, in scientific writing. Everything you show is shown for the first time, otherwise you wouldn’t bring it up. Everything that’s not shown ‘for the first time’ is mentioned in the state-of-the-art / introduction.

• Page 12 ‘aproximately 85%’ not an option – please remove.
	“most of” not an option – please remove. "performed by me” – exactly, you show what you did. Anything not done by you, has to be indicated as such, in the captions of the figure panels. Everything without explicit indication is assumed to be done by you. If that turns out not to be true, that’s a major issue. Hence, if in doubt it might be better to show less than more. 

% • page 13 “subsequent labeling” of tubulin – did you do that? Did we even use this kind of tubulin. Here at Biocev we buy it from cytoskeleton. in dresden it was made mostly by corina. 
% RE: I added ("performed by a laboratory assistant")

% • page 13/14 – position of page numbers changed. Please make consistent.
% RE: This follows from the double page layout and is consistent. Let me know if I should change the layout to single-page (I am planning to print double page)

% • page 14 – “Ase1 Set A experiments”, “Ase1 Set B experiments” this comes up here for the first time, without any explanation or intro. ‘Set’ could just as well be some stupid protein name, like ‘Pavarotti’. Can’t you just do it as in the paper. Just provide the method for the figures/panels? Sets are super confusing. Nobody, literally nobody, who could potentially read the thesis, can possibly understand what you are talking about. This is super hidden/convoluted, and when it comes up anywhere in the text, is always assumed to be already understood. – please remove.
% RE: Actually in the methods of the paper we have the terms set A/set B. I think the terms are helpful, but I now removed them from most places where they didn't seem particularly helpful. I also added the following sentence to the Methods: Experiments shown in Figures \ref{ase1c}F and \ref{ase2c}A,B were performed at the same experimental conditions as Set A experiments (with small differences dependent on the particular experiment as stated in the main text/figure captions).

• Page 15 TODO – please fix

% •page 16 4.4 ‘tau’ 4.4.1 ‘ase1’ please fix typo

% • page 16 purification – did you do that? (ok, maybe doesn’t matter, nobody will check, and it is good to have the method in the thesis)

% • page 18 move Lenka (brackets in first sentence, 4.5.1) to the place where you talk about neongreen purification, later in the section

% • page 18  ‘set B experiments’ – please remove, be specific: method for figures Xx and Xx.

% • page 19 Fig510A left panel, […] right panel

% • page 19 pinpoint determine the exact locations
% RE: You mean I should create a map of microtubules where I measured densities? That would seem excessive... We have this in the methods of the paper also and nobody asked for clarification.

• page 27 ‘largely performed by me’ – no need to indicate this. And, actually: Please remove! Everything (!) that is presented is assumed to be done by you! (not ‘largely’ but exclusively)! If that is not the case you have to indicate who did it. if this is done casually, not done correctly, we’ll be in trouble. For real. This is particularly true for Valeries thesis, they will cross-check. They will be meticulous. … but, it’s not a problem, IRL, because Valerie is showing two figures only. That’s all. However, you have to change the tone of your writing, you can’t be lax, or in any kind vague, about this. Still it’s easy, you just talk about the figures shown, everything else you don’t have to mention, or explain – as this would only create problems, without any kind of benefit.

% • page 33 ‘taucurveF’ please fix

% • page 34 – spaces missing in front of Fig.511A, and the next two

% • page35 “notably, with a different set of experiments (Set B …) we found the same result.” – makes it sound like you’re amazed that your findings are reproducible. I really would rephrase this. And I really would not do the Set A, Set B, thing. But then again, as Zdenek pointed just out (we talked about it) it’s your theis. You do whatever you want. All I’m saying is that both the ‘notably’ and the Set A, Set B are very (!) unfortunate, and that  I’d change this, if I were you. One other way to think about this: the text will be public, anyone can download it, and there’s no way to change it, once it is published.
% RE: Thanks for the warning, I rephrased this sentence to sound less surprised. However, I keep the "Set A/Set B" description because I think this is the more transparent way of presenting my findings.

% • page35 - Set A, Set B – maybe you could just state which conditions are used, and what’s the advantage of the second conditions, as we did in the paper.
% RE: There already is a sentence "Because the data for Set B experiments allowed for a more fine-grained analysis due to a higher framerate and a more pronounced accumulation effect [...]"

% • page 40 ‘we could not test for isolated MTs’ – no need to say that. This is just highlighing something that’s lacking. If someone asks, then you explain, if nobody asks, no need to highlight something that’s not there.

• page 42 – maybe more for discusion - how long is Ase1? The unstructured domains will be able to reach rather far (if fully extended). Wouldn’t the most straight forward mechanism/explanation for the reach (ase1 affecting the three terminal tubulins, not just the last one) … wouldn’t the easiest explanation be that detaching tubulins might be tethered by Ase1 proteins bound with their spectin domains to the MT lattice, while keeping the terminal tubulins on a tether, so that they maybe reattach (in equilibrium, high local concentration) before leaving completely?
	… just a thought. we might have talked about this before. you can add it or not, up to you. certainly not necessary at all. so it’s easier not to add it :)

% • page 46 “??F-H” and “Todo”– please fix

% • page 47 ‘high nucleation rate’ of what? Please add ‘tau islands’ “Todo”

• page 48  “TODO”

% • page 49  “shown in ??”, 6.3 please add, or remove, please avoid footnotes. 

• page 50  Please add an overall conclusion. This is somewhat important, but easy to do, hence still a ‘minor change’








