\chapter*{Abstract}
The cytoskeleton is a complex network of interlinking protein filaments which fulfills a diverse range of functions for the cell. One of the key type of protein filaments are microtubules, which themselves are present in a diverse set of cellular contexts. In axons, long, stable microtubule arrays within the microtubule shaft form highways for molecular motors, enabling them to efficiently transport cargo across the cell. \textit{In vitro} experiments, i.e., experiments where cellular components of interest are investigated outside of their cellular context, can help shed light on potential microscopic mechanisms underlying and giving rise to observed macroscopic phenomena. The results of the \textit{in vitro} experiments conducted for this thesis contribute to our understanding how these arrays could potentially be stabilized and protected against microtubule-severing enzymes while other microtubule regions within the neuron can remain dynamic. In particular, the results presented here show how the microtubule-associated protein Tau, which preferentially locates to axonal shafts, cooperatively binds to microtubules, forming islands which are highly effective at protecting microtubules against severing. We also show that these Tau islands regulate the activity of the molecular motor kinesin-8, and that this motor in turn is also capable of disassembling Tau islands. In the second \textit{in vitro} case study presented in this thesis, we show that the microtubule-crosslinking protein Ase1 can selectively stabilize antiparallel microtubule arrays, as they are found during mitosis, against depolymerization from the microtubule ends. This case study also shows that Ase1 is being herded by depolymerizing microtubule ends, and that this phenomenon likely is related to the propensity of Ase1 to oppose microtubule depolymerization. Overall, this the results in this thesis contribute to our understanding of how distinct interaction patterns between microtubules and microtubule-associated proteins can give rise to macroscopic structures within the cell.